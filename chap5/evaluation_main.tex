% !TeX root = ../cs_thesis_main.tex
\chapter{Evaluation}
This chapters serves as the synthesis of the study, translating the computational findings into actionable governance strategies for the Municipality of Bacolod. It begins by summarizing the critical behavioral drivers identified in the Global Sensitivity Analysis, primarily the overwhelming dominance of the "Cost of Effort" ($c_{effort}$), which accounted for approximately 83\% of the variance in compliance rates. This finding validates the "Convenience Hypothesis," confirming that physical friction and logistical difficulty are the primary inhibitors of waste segregation, often overriding an individual's pro-environmental intent. The chapter will contrast this with the near-zero sensitivity of the Attitude ($w_a$) parameter, a result that mathematically quantifies the "Value-Action Gap". This evidence supports the conclusion that the municipality’s current reliance on Information, Education, and Communication (IEC) campaigns has reached a point of diminishing returns, as awareness alone is insufficient to drive behavior change when systemic barriers remain high.

Building on these findings, the chapter will discuss the strategic pivot required in local policy, moving away from the "Status Quo" education-centric approach toward the "Sequential Saturation" strategy discovered by the Deep Reinforcement Learning (DRL) agent. It will explain the mechanism of "Cultural Inertia" and the critical "Tipping Point" observed in the simulation, where social norms act as a force multiplier only after a barangay reaches approximately 70\% compliance. The discussion will highlight how the DRL agent successfully exploited this non-linear dynamic by concentrating resources to "lock in" high compliance in specific areas before reallocating funds, rather than spreading the budget thinly across all barangays. This validates the strategic use of extrinsic motivators—specifically enforcement and incentives—to artificially induce the initial behavioral shift necessary to trigger self-sustaining social pressure.

\section{Global Sensitivity Analysis}

To evaluate the structural validity and predictive reliability of the developed Agent-Based Model (ABM), a Global Sensitivity Analysis (GSA) using Sobol indices was executed. Unlike local sensitivity analysis, GSA accounts for the non-linear interactions between variables across the entire multi-dimensional input space. This stress-testing phase involved varying the five core behavioral parameters ($w_a, w_{sn}, w_{pbc}, c_{\text{effort}}, \text{decay}$) by $\pm 20\%$ to quantify their individual contributions ($S_1$) to the variance in global compliance rates.

\begin{figure}[htbp]
    \centering
    % Ensure the filename matches exactly (case-sensitive on Overleaf/Linux)
    \includegraphics[width=0.8\textwidth]{Sobol.png}
    \caption{Global Sensitivity Analysis Results illustrating First-Order Sobol Indices ($S_1$) for behavioral parameters.}
    \label{fig:sobol_analysis}
\end{figure}

\subsection{Dominance of Cost ($S_1 \approx 0.83$)}
The analysis reveals that the Cost of Effort ($c_{\text{effort}}$) is the primary driver of the model's output, accounting for approximately 83\% of the variance in household compliance. This result provides computational evidence for the ``Convenience Hypothesis,'' suggesting that the physical friction and perceived difficulty of segregation are the most significant barriers to policy success.

This finding is strongly aligned with \citet{yazawa2025}, who examined the ``act and reality'' of the Ecological Solid Waste Management Act (RA 9003) at the barangay level in the Philippines. Their empirical observations suggest that structural deficiencies---such as lack of accessible collection points---often override individual intent. By identifying $c_{\text{effort}}$ as the dominant parameter, the model accurately reflects the Philippine local government context where logistical ease dictates the feasibility of waste programs.

\subsection{Influence of Social Norms ($S_1 \approx 0.35$)}
The secondary driver, Social Norms ($w_{sn}$), indicates that communal behavior acts as a critical force multiplier. The sensitivity index suggests that individual decisions are not made in a vacuum but are heavily contingent on the perceived compliance of the neighborhood.

This validates the ``Cultural Inertia'' mechanism, where social pressure can either lock a community into a state of non-compliance or accelerate a ``tipping point'' toward collective segregation. This mirrors the incentive pattern discussed by \citet{Zheng2022}, which emphasizes that social influence is a vital component in constructing sustainable resident behavior classification models.

\subsection{Inefficacy of Attitude ($S_1 \approx 0$) and the ``Value-Action Gap''}
A significant revelation of the GSA is the near-zero sensitivity of the Attitude ($w_a$) parameter. While traditional policy often focuses on Information, Education, and Communication (IEC) campaigns to change mindsets, the model demonstrates that ``environmental awareness'' alone is insufficient to drive behavior change when systemic barriers remain high.

This phenomenon, known as the Value-Action Gap, is well-documented in environmental psychology and supported by the findings of \citet{Zhao2022}. Their research on waste sorting in developing countries suggests that extrinsic motivators (such as the DRL agent's enforcement and incentive strategies) are significantly more effective than intrinsic motivators (education) in bridging the gap between what citizens value and how they actually behave. Consequently, the model's behavior confirms that the DRL agent’s preference for enforcement over education is not a computational artifact, but a reflection of optimal utility maximization under real-world constraints.

\section{The Determinants of Compliance}
The findings from the evaluation phase provide a critical computational justification for the strategic trajectory of the Deep Reinforcement Learning (DRL) agent, specifically its preference for the ``Sequential Saturation'' strategy over the existing education-centric ``Status Quo.'' By examining the behavioral sensitivity of the model, this discussion bridges the gap between the algorithmic outputs of the AI and the socioeconomic realities of waste management in the Municipality of Bacolod. The dominance of specific parameters suggests that the DRL agent did not merely find a numerical shortcut to high compliance; rather, it successfully identified and exploited the underlying behavioral drivers that dictate how households interact with municipal policy.

\subsection{The ``Convenience Barrier'' and the Dominance of Cost}
The overwhelming sensitivity of the model to the Cost of Effort ($c_{\text{effort}}$), which accounts for approximately 83\% of the variance in global compliance rates ($S_1 \approx 0.83$), provides mathematical validation for what is known in environmental sociology as the ``Convenience Hypothesis.'' Within the framework of this study, ``Cost'' is not defined solely by financial penalties or fines; it represents the total ``friction'' of the activity, including temporal, physical, and cognitive effort.

In the context of Bacolod, Lanao del Norte, this sensitivity implies that if the segregation process is characterized by high friction—such as the requirement to purchase specific color-coded liners, the need to traverse significant distances to communal collection points, or the time-intensive nature of sorting organic from inorganic waste—household agents will default to non-compliance regardless of their pro-environmental beliefs. This finding is strongly aligned with the empirical work of \citet{Yazawa2025}, who examined the ``act and reality'' of the Ecological Solid Waste Management Act (RA 9003) at the barangay level in the Philippines. Their research suggests that structural deficiencies and logistical failures often override individual intent, rendering traditional policy ineffective. Consequently, the DRL agent’s aggressive funding of Enforcement (to increase the cost of non-compliance) and Incentives (to offset the cost of compliance) represents a rational response to this ``Convenience Barrier.'' The AI learned that it could not simply ``educate'' away the physical friction of the waste system; it had to fundamentally alter the individual utility calculus by making non-compliance prohibitively expensive or compliance logistically rewarding.

\subsection{The ``Value-Action Gap'': Why Education Exhibited Diminishing Returns}
A significant revelation of the sensitivity analysis is the near-zero sensitivity of the Attitude ($w_a$) parameter. While traditional municipal policies often focus heavily on Information, Education, and Communication (IEC) campaigns to shift public mindsets, the model demonstrates that ``environmental awareness'' alone is an insufficient driver for behavior change when systemic barriers remain high. This phenomenon is a computational realization of the ``Value-Action Gap''—the well-documented discrepancy in environmental psychology between what individuals say they value and what they actually do.

The model reality suggests that even when households were initialized with high attitude scores—simulating the success of an intensive IEC campaign—they failed to comply if the cost of effort remained high or if the regulatory presence was negligible. This confirms that while awareness may be a necessary prerequisite, it has a low ceiling of effectiveness in the absence of structural support. As noted by \citet{Zhao2022} in their research on waste sorting in developing nations, extrinsic motivators such as incentives and regulatory pressure are significantly more effective than intrinsic motivators (education) in promoting consistent household participation. For the Municipality of Bacolod, these results suggest a state of diminishing returns on education. Residents likely understand the ecological importance of segregation, but the current policy framework fails to address the gap where that knowledge should turn into action. The DRL agent's decision to pivot away from IEC funding is therefore not a dismissal of education, but a strategic recognition that the most effective lever for immediate compliance lies in structural and extrinsic motivators.

\subsection{Cultural Inertia and the ``Tipping Point''}
The significant sensitivity to Social Norms ($w_{sn}$, $S_1 \approx 0.35$) validates the ``King of the Hill'' or ``Sequential Saturation'' strategy favored by the AI. Social Norms function as a non-linear behavioral multiplier within the social fabric of the barangays. The mechanism operates on the principle of communal visibility: when compliance is low (e.g., $<10\%$), the social norm reinforces non-compliance as the acceptable communal standard, creating a state of ``Cultural Inertia'' that resists change.

However, the DRL agent identified a critical threshold—a ``Tipping Point''—where social influence flips from a barrier to a facilitator. Once enforcement and incentives push a barangay’s compliance past a critical mass (identified in the simulation as approximately 70\%), the Social Norm parameter begins to act as a reinforcement mechanism. In this state, the pressure to conform to the now-visible majority behavior sustains high compliance even as the agent reallocates resources to other areas. This explains the ``Graduation Effect'' observed in the longitudinal data of Chapter 4, particularly in barangays like Mati, where compliance remained stable after the withdrawal of heavy funding. This confirms the assertions of \citet{Zheng2022} regarding the construction of sustainable incentive patterns; the ultimate goal of the LGU should not be perpetual, expensive enforcement, but rather the strategic use of resources to reach the cultural tipping point where the behavior becomes self-sustaining through peer pressure and community standard-setting. This ``lock-in'' effect is the key to long-term policy sustainability without permanent financial strain on the municipal budget.

\section{Summary}
The evaluation of the Agent-Based Model through Global Sensitivity Analysis (GSA) and behavioral determinant mapping provides a robust computational foundation for the DRL agent's optimized policies. By transitioning from a ``Status Quo'' approach to a ``Sequential Saturation'' strategy, the model demonstrates how municipal resources can be leveraged more effectively by targeting the underlying mechanics of household decision-making. These findings suggest that successful waste management policy in the local context is less about shifting public opinion and more about strategically reducing the structural friction inherent in the waste disposal process.

The primary discovery of this evaluation is the dominance of convenience as a behavioral driver. With the Cost of Effort ($c_{\text{effort}}$) accounting for 83\% of output variance, the study confirms that logistical friction is the single greatest inhibitor of waste segregation in the Municipality of Bacolod. Policies that do not actively reduce this ``cost''—either through direct infrastructure improvements or compensatory incentives—are statistically likely to fail regardless of the level of public support. This underscores a critical ``Convenience Hypothesis'' where the physical and temporal burden of sorting waste outweighs any intrinsic motivation.

Furthermore, the results highlight a distinct ``Value-Action Gap'' characterized by the near-zero sensitivity of the Attitude ($w_a$) parameter. This mathematically illustrates that while Information, Education, and Communication (IEC) campaigns are foundational, they have reached a point of diminishing returns. The DRL agent's strategic shift toward extrinsic motivators, such as enforcement and subsidies, is thus justified; awareness alone cannot bridge the gap between intention and action when systemic barriers remain high. By prioritizing structural interventions over repetitive education, the agent targets the levers that actually trigger behavioral change.

Finally, the model identifies social multipliers as the key to long-term sustainability. The significant influence of Social Norms ($w_{sn} \approx 0.35$) provides the mechanism for the ``Graduation Effect'' observed in the simulation. By focusing resources on achieving a 70\% compliance tipping point, the Local Government Unit (LGU) can trigger self-sustaining cultural shifts. Once this threshold is crossed, the social pressure to conform maintains high compliance, allowing for the eventual reallocation of funds without a collapse in segregation rates. In conclusion, the DRL agent’s preference for localized enforcement and incentives is a sophisticated response to the socio-technical barriers in Philippine waste management, offering a data-driven roadmap toward a resilient compliance ecosystem.