% !TeX root = ../cs_thesis_main.tex
\chapter{Introduction}

The effective management of solid waste begins with a clear and functional definition of what constitutes ``waste.'' At the most fundamental level, solid waste encompasses any discarded materials that are no longer required by their owner or user, including refuse, trash, and garbage \citep{UNEP2024}. This broad definition is not limited to materials that are physically solid; it can also include liquids, semi-solids, or contained gaseous materials that are discarded. The global waste stream is highly complex and is often categorized by origin (e.g., municipal, industrial, agricultural), type (e.g., electrical and electronic waste, or e-waste), and character (e.g., hazardous waste) to facilitate targeted management strategies \citep{WHO2024}.

Within this broad landscape, Municipal Solid Waste (MSW) represents a critical and highly visible subset. MSW is generally defined as waste generated from residential and commercial sources and is typically managed by local governments \citep{Kaza2021}. This category includes household garbage, similar waste from commercial establishments and institutions, yard trimmings, and street sweepings, but it typically excludes waste from municipal sewerage networks, industrial processes, and major construction and demolition activities \citep{Kaza2021}. The effective management of MSW necessitates Solid Waste Segregation at Source (SWS), the required process of separating waste components—often into biodegradable, non-biodegradable, and recyclable streams—at the exact point of generation, typically the household or commercial establishment \citep{WorldBank2022}. This initial sorting is mandated globally as the foundational step for resource recovery, recycling, and composting \citep{WorldBank2022}. Academic literature indicates that the effectiveness of SWS is highly contingent upon behavioral factors and the provision of adequate infrastructure, such as appropriate bins and bags, that would simplify the separation process for households.

The localized challenge of SWS is magnified by an escalating global solid waste crisis, making policy optimization a matter of urgent international significance. The volume of municipal solid waste (MSW) generated worldwide is currently over 2.0 billion metric tons annually \citep{Kaza2021}. Projections from the World Bank indicate that without drastic intervention, this figure is set to surge by 73 percent, reaching nearly 3.9 billion metric tons by 2050 \citep{WorldBank2025, WorldBank2021}. This massive growth rate, driven primarily by increasing urbanization and consumption patterns in developing economies, places extreme pressure on municipal services globally. The failure to manage this escalating volume effectively incurs significant financial penalties. While the direct global cost of waste management in 2020 was an estimated 252 billion USD, the total cost—including the hidden costs associated with environmental pollution, climate change externalities (such as methane emissions), and detrimental public health outcomes—rose to an estimated 361 billion USD \citep{UNEP2024}. Should current trends persist, this comprehensive annual global cost is projected to almost double, reaching a staggering 640.3 billion USD by 2050 \citep{UNEP2024}. Globally, an estimated 33 percent of waste is improperly managed (often openly dumped or burned) \citep{Kaza2021}.

This global crisis is sharply realized in specific localized contexts. The definition of solid waste, however, is not merely descriptive; it is a prescriptive tool that shapes policy and regulation. In the United States, for example, the Resource Conservation and Recovery Act (RCRA) provides a highly legalistic framework where a material must first meet the definition of "solid waste" before it can be classified as "hazardous waste" and subjected to stricter controls (\cite{WHO2024}).

Similarly, the Philippines operates under a specific legal framework established by Republic Act No. 9003, the Ecological Solid Waste Management Act of 2000. Under RA 9003, "solid waste" is defined as all discarded household, commercial, non-hazardous institutional and industrial waste, street sweepings, construction debris, and agricultural waste \citep{RA9003}. Critically, the Act explicitly excludes certain materials from this definition, including waste identified as hazardous, infectious waste from hospitals, and waste from mining activities, which are presumably managed under separate regulations \citep{RA9003}. The Act further defines "municipal waste" as the combination of domestic, commercial, institutional, and industrial wastes and street litters generated within the jurisdiction of local government units (LGUs) \citep{RA9003}.

The management of Municipal Solid Waste (MSW) remains a critical environmental and public health challenge in the Philippines. Despite the landmark Ecological Solid Waste Management Act of 2000 (RA 9003), which mandates segregation at source, compliance at the household level is inconsistent and often at a "less extent" \citep{Carpio2025}. This failure is attributed to a complex interplay of factors, including limited LGU resources, weak enforcement, insufficient public awareness, and a critical lack of functional infrastructure like Materials Recovery Facilities (MRFs) \citep{Espino2025}. Recent data from the Philippine Statistics Authority (PSA) reveals a crucial qualitative failure in the solid waste management system. Between 2023 and 2024, the Philippines saw significant quantitative growth in mandated facilities: the number of Material Recovery Facilities (MRFs) increased by 8.7 percent, reaching 12,855 nationwide \citep{PSA2024}. Simultaneously, however, the number of reported illegal dumpsites surged by a massive 84 percent, climbing from 43\% in 2023 to 79\% in 2024 \citep{PSA2024}. This contradictory trend constitutes the MRF paradox, demonstrating that the primary operational bottleneck is the failure of source waste segregation (SWS), not merely a lack of infrastructure investment \citep{Carpio2025, WorldBank2022}.

Local Government Units in the Philippines currently operate under a system of trial-and-error, experimenting with a range of behavioral policy instruments to boost SWS compliance. These approaches vary widely, from strictly punitive measures such as the "No Segregation, No Collection" (NSNC) fine policies \citep{Collado2024}, to various incentive-based programs like the "Basura Store," which allows residents to exchange recyclable wastes for essential goods such as rice or canned items \citep{Camarillo2021}. However, the effectiveness and sustainability of these policies are highly variable and remain poorly understood across diverse socioeconomic settings. The current lack of a standardized, evidence-based policy framework leads to suboptimal waste outcomes and wasted public resources. In this study, the researchers will systematically quantify and determine the optimal settings for policy instruments—specifically incentives and punitive measures—required to maximize sustained household solid waste segregation compliance within a representative Philippine Local Government Unit (LGU), taking into account local socioeconomic determinants.

\section{Background of the Study} 

The Municipality of Bacolod is a 4th Class Municipality located in the Province of Lanao del Norte, Northern Mindanao (Region X), Philippines. It is a coastal town characterized by a blend of urban and rural barangays. This distinct classification provides a valuable context for solid waste management (SWM), as the challenges faced by municipalities differ significantly from those of highly urbanized cities—often involving more constrained financial resources, logistical difficulties in collection across disparate areas, and the preservation of natural resources like coastlines and agricultural land.

According to the 2024 Census of Population and Housing (CPH), the Municipality of Bacolod has a total population of 24,963 inhabitants \citep{PSA2024}. The daily waste generated by this population, while lower in absolute volume than a large city, still requires a systematic and effective management system to prevent environmental degradation, especially given its coastal location along Iligan Bay.

\begin{figure}[h] % [h] suggests placing it 'here'
    \centering
    % 1/8 converted to decimal is 0.125
    \includegraphics[width=0.6\linewidth]{map_bacolod.png}
    \centering
    \caption{Bacolod in highlight Lanao del Norte Map}
    \label{fig:map_bacolod}
\end{figure}

The foundation of SWM in the Philippines is the Republic Act No. 9003, also known as the Ecological Solid Waste Management Act of 2000. This law mandates all Local Government Units (LGUs)—including the Municipality of Bacolod—to implement an ecological and comprehensive SWM program. Key legal requirements of RA 9003 that guide and challenge Bacolod include:

\begin{itemize}
    \item Mandatory separation of waste into biodegradable, recyclable, and residual categories at the household and establishment level.
    \item Functional facilities in every barangay or cluster of barangays for final sorting, composting, and recycling.
    \item Prohibiting the use of open dumpsites and mandating the development of compliant sanitary landfills or alternative technologies.
    \item  Requiring the LGU to craft and implement a comprehensive, long-term plan approved by the National Solid Waste Management Commission (NSWMC).
\end{itemize}

To enforce this, Bacolod Lanao del Norte operates under Municipal Ordinance No. 2018-05, also known as the "Ecological Solid Waste Management of 2018" (refer to Appendix \ref{appendix:esw_ordinance}). This local legislation formalizes the mandates of RA 9003, detailing the mandatory source segregation using a four-color-coded bin system (Green for biodegradable, Black for residual, Blue for recyclable, and Red for toxic) (refer to Appendix \ref{appendix:esw_ordinance}, Sec. 8). It also explicitly prohibits single-use plastics and polystyrene (Styrofoam) (refer to Appendix\ref{appendix:esw_ordinance}, Sec. 13) and outlines a schedule of fines for individual and establishment non-compliance (refer to Appendix \ref{appendix:esw_ordinance}, Sec. 15).

A core provision of this policy is the mandatory "no segregation, no collection" rule, which is explicitly stated in the ordinance as "Unsegregated waste shall not be collected" (refer to Appendix \ref{appendix:esw_ordinance}, Sec. 8). The Municipal Environment and Natural Resources Office (MENRO) confirms this, clarifying that the LGU is responsible for collecting waste from the Barangays, but only if that waste has already been properly segregated at the barangay level (refer to Appendix \ref{appendix:menro_interview}).

This thesis aims to determine the optimized policy so that citizens comply with these SWM policies by focusing on the performance of two critical institutional functions: policy implementation and citizen behaviors towards household waste segregation.

The research explores the dynamics between local policy formulation and on-the-ground reality, particularly the persistent challenge of citizen attitude and compliance towards waste segregation. Despite the existence of the ordinance, the MENRO Head estimates the segregation rate at the household source to be only about 10\% (refer to Appendix \ref{appendix:menro_interview}). The successful implementation of RA 9003 ultimately hinges on the active and consistent participation of every household—a behavioral factor that the environmental enforcement structure is tasked to influence.

However, this enforcement structure faces severe limitations. The MENRO identifies its main challenges as a lack of budget and manpower (refer to Appendix \ref{appendix:menro_interview}). The SWM program budget of approximately 1.5 million pesos is described as "kulang" (insufficient) to cover all SWM activities, biodiversity projects, and collection, which in turn prevents the hiring of more enforcers and leaves many plans on an aspirational planning due to budgetary deficits. (refer to Appendix \ref{appendix:menro_interview}). This resource gap creates a difficult enforcement dilemma: the MENRO Head notes that if the ordinance were strictly enforced, "all households would be penalized," which is considered unfeasible. Consequently, the LGU must balance limited enforcement (e.g., citation tickets, "Eco-warriors") with continuous Information, Education, and Communication (IEC) campaigns, viewing the primary obstacles as "social norms, acceptance, and behavioral constraints" (refer to Appendix \ref{appendix:menro_interview}).

The Municipality of Bacolod, Lanao del Norte, has been strategically selected as the research locale due to its geographic accessibility and the established collaborative relationships with the key municipal government offices, as evidenced by the initial qualitative interviews. This logistical advantage is crucial for the research methodology. By focusing on a 4th Class Municipality, this study offers valuable insights into how smaller LGUs, facing confirmed budgetary and manpower constraints (refer to Appendix \ref{appendix:menro_interview}) and significant logistical hurdles—such as collection from "very far" inland barangays (refer to  Appendix \ref{appendix:menro_interview})—interpret and strive for compliance with stringent national environmental policies.

This study aims to systematically quantify and determine the optimal settings for policy instruments, specifically incentives, punitive, information and educational campaign measures, required to maximize sustained household solid waste segregation compliance within the Municipality of Bacolod, Lanao del Norte, by taking into account local socioeconomic determinants and budget-constraints.

\section{Statement of the Problem}
This study aims to systematically quantify and determine the optimal settings for policy instruments, specifically incentive, punitive, information and educational campaign measures, required to maximize sustained household solid waste segregation compliance within the Municipality of Bacolod, Lanao del Norte, by taking into account local socioeconomic determinants and budget-constraints.

% \noindent removes the indentation for this specific line
\noindent This study seeks to answer the following specific questions:

\begin{enumerate}
    \item How do variations in the synthesized household behavioral parameters (e.g., the relative weight of Subjective Norms vs. Perceived Behavioral Control, derived from literature and LGU records) affect the stability and efficacy of policy outcomes within the Agent-Based Model?
    \item What is the optimal long-term resource allocation ratio among the three policy levers (monetary incentives, punitive enforcement, and educational campaigns) that maximizes compliance per peso spent, as determined by the Reinforcement Learning agent?
    \item Which dynamic policy strategy yields the highest overall compliance and cost-benefit ratio for the LGU while strictly adhering to the defined annual budget constraint?
\end{enumerate}

\section{Research Objectives}

The primary objective of this study is to develop and apply a coupled Agent-Based Model (ABM) and Deep Reinforcement Learning (DRL) framework to determine the optimal, budget-constrained allocation of resources across policy levers for maximizing household solid waste segregation compliance in the Municipality of Bacolod.

\noindent Specific objectives are:

\begin{enumerate}
    \item To conduct a comprehensive synthesis of academic literature and utilize contextual financial and operational data from the Philippine Statistics Authority and LGUs records, including interviews with key implementing officers, to rigorously parameterize the ABM.
    \item To develop a Multi-Level Agent-Based Model where household agent behavior is governed by a utility function incorporating Theory of Planned Behavior constructs and socioeconomic variables, and where policy levers dynamically update behavioral constructs.
    \item To integrate a Reinforcement Learning algorithm that enables the LGU agent to autonomously learn the optimal policy (allocating funds among incentives, enforcement staff, and education campaign) that maximizes a composite reward function balancing compliance and financial cost, while adhering to a defined budget constraint.
    \item To simulate and validate the efficacy and cost-effectiveness of budget allocation strategies (Pure Incentive, Pure Penalty, Pure Information Education Campaign, and Hybrid regimes) and provide actionable, data-driven recommendations on the optimal resource mix for the LGU enforcing RA 9003.
\end{enumerate}

\section{Significance of the Study}

\setlength{\parindent}{2em} 

The findings of this research are expected to yield significant contributions across academic, practical, and policy domains.

From an Academic Research perspective, this work contributes to the interdisciplinary fields of environmental science, computational social science, and public policy. It advances the application of Agent-Based Modeling by integrating a robust psychological framework (Theory of Planned Behavior) with Reinforcement Learning for policy optimization—a novel computational approach in the context of Philippine solid waste management. Furthermore, the study provides a validated and parameterized model that can be adapted and reapplied for other behavioral and policy studies focused on resource and behavioral challenges in developing countries. While traditional optimization in waste management often relies on static linear programming or heuristic methods, these approaches fail to capture the non-linear and adaptive nature of household behavior \citep{TianReview2024}. This study advances the field by employing Deep Reinforcement Learning (DRL), specifically utilizing Deep Neural Networks (DNNs) as function approximators. Unlike standard tabular RL, which struggles with the 'curse of dimensionality,' DRL enables the LGU agent to process high-dimensional state spaces—such as varying compliance rates across seven distinct barangays—to autonomously discover complex, adaptive policy strategies \citep{Dey2025, HaMinh2025}. Furthermore, by formalizing the simulation environment as a Markov Decision Process (MDP), this research demonstrates how Agent-Based Models can serve as robust data generators for training AI policies in the absence of historical datasets \citep{Kompella2020, Jimenez2025}

To Local Government Units (LGUs), the study directly addresses the operational challenges of policy implementation by providing a powerful, low-risk decision-support tool. Instead of relying on costly and time-consuming real-world trials, policymakers can use the developed Agent-Based Modeling and Deep Reinforcement Learning (ABM-DRL) framework to test and identify the most cost-effective policy mix (incentives, fines, or hybrid) tailored for their specific community context. The resultant data-driven recommendations, such as an optimal fine-to-incentive ratio, are actionable and aim to lead to more effective waste management, better allocation of public funds, and ultimately, higher compliance with RA 9003.

On a broader scale, the successful implementation of the study's recommendations enhances National Policy and Environmental Sustainability. By improving segregation at source, the research contributes to crucial downstream waste management benefits: a reduced volume of waste going to landfills, increased recovery of recyclables, and the resulting promotion of a circular economy. This enhanced system ultimately leads to improved public health, environmental protection, and supports national climate change mitigation goals through the reduction of methane emissions from landfills.

\section{Scope and Limitations}
\setlength{\parindent}{2em}

This study is bounded by specific constraints concerning its geographical focus, methodological framework, and data utilization strategies.

The computational model is explicitly contextualized within the Municipality of Bacolod, Lanao del Norte, simulating the multi-level governance dynamics between the municipal Local Government Unit (LGU) and its constituent barangays. The study utilizes the municipality's local Solid Waste Management Ordinance to establish baseline structures for punitive and incentive-based policies. Crucially, the simulation is operationally limited to the seven (7) barangays currently covered by the municipal waste collection system. The remaining nine barangays are excluded from the scope due to logistical inaccessibility and their location outside the current service coverage area. Consequently, findings regarding optimal policy parameters are most directly applicable to LGUs sharing similar socioeconomic profiles and logistical constraints.

As an Agent-Based Model (ABM), this research serves as a necessary abstraction of reality. The scope is strictly focused on household solid waste segregation at the source. It does not model the entire solid waste management value chain—such as final disposal, sanitary landfill management, or the technical operations of Material Recovery Facilities (MRFs)—except where infrastructure availability directly influences the residents' Perceived Behavioral Control. The LGU agent’s strategic space is limited to adjusting three specific policy levers: the magnitude of monetary incentives, the severity of punitive fines (modeled as enforcement costs), and the intensity of educational campaigns. The optimization process excludes operational logistics, such as waste collection routing or fleet management. Furthermore, the LGU agent's decision-making is strictly constrained by a fixed, simulated annual operating budget.

The study is delimited to data synthesis from secondary sources and a rigorous literature review; All behavioral parameters required for the model (e.g., Theory of Planned Behavior construct weights) will be derived solely from a meta-analysis and synthesis of existing academic studies relevant to SWM in developing countries. Specifically, baseline agent values for Knowledge, Attitude, and Practice (KAP) will be calibrated using regional empirical data from \cite{Paigalan2025}, which characterizes the KAP profile of riverside barangays in Northern Mindanao, serving as a high-fidelity proxy for the coastal Municipality of Bacolod. Therefore, the model's validity is conditional on the transferability and representativeness of these synthesized parameters. Finally, the simulation relies solely on the TPB as the cognitive framework for household agents, and the policy instruments are restricted to direct monetary incentives, punitive fines, information educational campaigns, and hybrid combinations thereof.

Crucially, this study clarifies the operational definition of \textit{``policy optimization.''} The research does not propose the drafting of a new legislative ordinance to replace Municipal Ordinance No. 2018-05 (reder to \ref{appendix:esw_ordinance}). Instead, it focuses on the \textit{executive implementation of the existing policy}. In this context, the study distinguishes between three policy states: the \textit{Original Policy} which is the current status quo, \textit{Modified Policies}that are experimental regimes, and the \textit{New Policy} that is the optimized, adaptive resource allocation strategy generated by the Deep Reinforcement Learning agent.

Regarding enforcement integrity, the model operates under the assumption of honest agent interactions. While preliminary interviews in Barangay Liangan East (refer to Appendix \ref{appendix:liangan_interview}) suggest the existence of informal transactions—such as residents tipping collectors to accept unsegregated waste—this study focuses on optimizing official policy levers. The modeling of systemic corruption or bribery introduces game-theoretic complexities that are outside the scope of this research; therefore, informal tipping and enforcement bypass mechanisms are excluded from the simulation.

\section{Document Structure}

\blindtext
