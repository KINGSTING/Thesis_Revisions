\chapter{Interview Transciptions}

Presented in this chapter are the full transcripts of semi-structured interviews conducted with key stakeholders in the Municipality of Bacolod, including the Municipal Environment and Natural Resources Officer (MENRO) and Barangay officials. These discussions provided the essential ground-truth data required to contextualize the study's investigation into local solid waste management policies.

\section{Municipal Environment and Natural Resources Officer (MENRO)}
\label{appendix:menro_interview}

\noindent
\textbf{Resource Person:} Engr. Archer M. Zamora \\
\textbf{Position:} MENRO Head, LGU-Bacolod \\
\textbf{Topic:} SWM Policy Implementation, Budget, and Challenges

\vspace{0.5cm}

%---------------------------------------------------------
%   THEME 1: POLICY AND ROLES
%---------------------------------------------------------
\noindent \textbf{\large Policy and Roles}
\vspace{0.2cm}

\noindent \textbf{Q: What are the primary laws and ordinances guiding your solid waste management (SWM) implementation?} \\
\textbf{A:} We are focused on implementing Republic Act 9003 (The Ecological Solid Waste Management Act). This includes enforcing local ordinances for the segregation of waste and the prohibition of single-use plastics.

\vspace{0.3cm}

\noindent \textbf{Q: How are SWM responsibilities divided between the LGU and the Barangays?} \\
\textbf{A:} The Barangay is the frontline of implementation. Their primary role is the collection of garbage at their level and the operation of their local Materials Recovery Facility (MRF). The LGU (Municipal) level is then responsible for collecting the waste from the Barangays, but we only collect waste that has already been properly segregated.

\vspace{0.5cm}

%---------------------------------------------------------
%   THEME 2: INCENTIVES AND PENALTIES
%---------------------------------------------------------
\noindent \textbf{\large Incentives and Penalties (The ``Carrot and Stick'')}
\vspace{0.2cm}

\noindent \textbf{Q: What incentives, or ``carrot'' approaches, do you use to encourage Barangays and households to comply?} \\
\textbf{A:} We use several ``carrot-style'' incentives. This includes:
\begin{itemize}
    \item Recognition for the best-implementing Barangays.
    \item Programs like Ecobrick exchange for goods.
    \item Before the pandemic, we held competitions, but this was stopped due to budget constraints.
\end{itemize}

\noindent \textbf{Q: What penalties, or ``stick'' approaches, are used for enforcement?} \\
\textbf{A:} For enforcement, the MENRO has an inspection team. We issue citation tickets for penalties to violators. We also utilize CSU (Civil Security Unit) or ``Eco-warrior'' enforcers to monitor compliance.

\vspace{0.5cm}

%---------------------------------------------------------
%   THEME 3: BUDGET AND MANPOWER
%---------------------------------------------------------
\noindent \textbf{\large Budget and Manpower}
\vspace{0.2cm}

\noindent \textbf{Q: What is the budget for your SWM programs?} \\
\textbf{A:} Our program budget is approximately 1.5 million pesos. This budget has to cover collection, biodiversity projects, and all Solid Waste Management activities.

\vspace{0.3cm}

\noindent \textbf{Q: What are the main challenges you face with enforcement and resources?} \\
\textbf{A:} Our main challenges are budget and manpower.
\begin{itemize}
    \item \textbf{Manpower:} We have a significant lack of staff for enforcement.
    \item \textbf{Budget:} We cannot employ more enforcers because of budget constraints. Frankly, the budget is \textit{kulang} (insufficient). Many of our plans are on a ``wishlist'' because of these limited funds.
\end{itemize}

\vspace{0.5cm}

%---------------------------------------------------------
%   THEME 4: AWARENESS AND BEHAVIOR
%---------------------------------------------------------
\noindent \textbf{\large Awareness and Behavioral Challenges}
\vspace{0.2cm}

\noindent \textbf{Q: How do you handle Information, Education, and Communication (IEC) campaigns?} \\
\textbf{A:} We run continuous sanitation and IEC campaigns. Our most effective tool is the local radio station (101.3 Grace Covenant FM), and we consistently provide reminders during every assembly. This requires a specific budget for radio advertising.

\vspace{0.3cm}

\noindent \textbf{Q: What is the biggest obstacle to successful waste segregation?} \\
\textbf{A:} The biggest obstacles are behavioral and cultural. We produce waste every day. Even if we have a complete ordinance, it will not be successful if we don't get cooperation from the people. The main problems are social norms, acceptance, and behavioral constraints.

\vspace{0.3cm}

\noindent \textbf{Q: What is the current rate of segregation at the source?} \\
\textbf{A:} We estimate the segregation rate at the household source is only about 10\%. However, establishments (businesses) are generally compliant and do segregate their waste.

\vspace{0.3cm}

\noindent \textbf{Q: If compliance is low, why not strictly penalize all non-compliant households?} \\
\textbf{A:} The problem is acceptance. If we were to be extremely strict right now, all households would be penalized, which isn't feasible. We must balance enforcement with continuous awareness.

\vspace{0.5cm}

%---------------------------------------------------------
%   THEME 5: ACCOUNTABILITY AND LOGISTICS
%---------------------------------------------------------
\noindent \textbf{\large Accountability and Logistics}
\vspace{0.2cm}

\noindent \textbf{Q: How are Barangay officials held accountable for implementing SWM?} \\
\textbf{A:} Accountability is handled in several ways:
\begin{itemize}
    \item \textbf{Monitoring:} The DENR (Department of Environment and Natural Resources) monitors compliance.
    \item \textbf{Council Meetings:} We hold mandatory quarterly meetings with the Solid Waste Management Council (SWMC), which includes Barangay Officials and the LGU.
    \item \textbf{Sanctions:} As provided by law, Barangay officials may be suspended for failure to comply. However, we prefer to focus on awareness and reminders because we view them as partners.
\end{itemize}

\vspace{0.3cm}

\noindent \textbf{Q: What is the long-term strategy for improving these numbers?} \\
\textbf{A:} We are following our 10-year Solid Waste Management plan, which focuses on consistent reminders and awareness campaigns.

\vspace{0.3cm}

\noindent \textbf{Q: How do you manage collection for inland Barangays that are hard to reach?} \\
\textbf{A:} The inland Barangays have their own MRF and segregation facilities. Collection is a major issue due to accessibility—they are \textit{layo na kaayo} (very far). They often have to use their own initiative, such as using a multipurpose vehicle, to deliver their segregated garbage to the collection points.

\vspace{0.3cm}

\noindent \textbf{Q: Which specific Barangays does the LGU collect segregated waste from?} \\
\textbf{A:} The LGU currently collects from 7 Barangays, these are:
\begin{enumerate}
    \item Liangan East
    \item Esperanza
    \item Poblacion
    \item Binuni
    \item Demologan
    \item Mati
    \item Babalaya
\end{enumerate}

\section{Barangay Liangan East}
\label{appendix:liangan_interview}

\noindent
\textbf{Resource Persons:} \\
1. Hon. Rufo Palangan Lumacad (Barangay Captain) \\
2. Ms. Analyn J. Eltagon (Barangay Secretary) \\
\textbf{Location:} Brgy. Liangan East, Bacolod, Lanao del Norte

\vspace{0.5cm}

%---------------------------------------------------------
%   THEME 1: DEMOGRAPHICS AND PROFILE
%---------------------------------------------------------
\noindent \textbf{\large Part 1: Demographics and Profile}
\vspace{0.2cm}

\noindent \textbf{Q: (To Secretary) What is the total population and number of households in Barangay Liangan East?} \\
\textbf{A:} We have a total population of 2,198 individuals and a total of 608 households.

\vspace{0.3cm}

\noindent \textbf{Q: What would you say is the average household size?} \\
\textbf{A:} The average household size is about 4 members.

\vspace{0.3cm}

\noindent \textbf{Q: How would you describe the general income level of the residents?} \\
\textbf{A:} The general income level is middle class.

\vspace{0.3cm}

\noindent \textbf{Q: How many barangay officials and BPAT members do you have?} \\
\textbf{A:} We have 22 Barangay Officials and 10 Barangay Peacekeeping Action Team (BPAT) members.

\vspace{0.5cm}

%---------------------------------------------------------
%   THEME 2: POLICIES AND INFORMATION
%---------------------------------------------------------
\noindent \textbf{\large Part 2: Policies and Information Campaigns}
\vspace{0.2cm}

\noindent \textbf{Q: What official policies does the barangay have for solid waste management?} \\
\textbf{A:} We follow the Executive Order and the Municipal Ordinance.

\vspace{0.3cm}

\noindent \textbf{Q: How do you communicate these policies to the residents?} \\
\textbf{A:} We use Information Communication Campaigns and the Barangay Assembly. We also hold meetings by Purok, led by the Captain and Kagawad.

\vspace{0.3cm}

\noindent \textbf{Q: What is your single most important rule regarding household garbage collection?} \\
\textbf{A:} ``No Segregation, No Collection'' policy.

\vspace{0.5cm}

%---------------------------------------------------------
%   THEME 3: ENFORCEMENT AND COMPLIANCE
%---------------------------------------------------------
\noindent \textbf{\large Part 3: Enforcement and Community Compliance}
\vspace{0.2cm}

\noindent \textbf{Q: Who is responsible for enforcing the ``No Segregation, No Collection'' policy on the ground?} \\
\textbf{A:} The BPAT and Barangay Officials are responsible for enforcement.

\vspace{0.3cm}

\noindent \textbf{Q: What is the process when a household is non-compliant? Does the barangay issue a ticket?} \\
\textbf{A:} The Barangay does not issue tickets directly. We only report the violators. The MENRO is the one handling the citation tickets. The penalty comes from a summon by the MENRO and LGU based on our reports.

\vspace{0.3cm}

\noindent \textbf{Q: What is the current compliance rate for waste segregation among the households?} \\
\textbf{A:} It is around 60-70\% compliance.

\vspace{0.3cm}

\noindent \textbf{Q: Have you noticed this compliance changing over time?} \\
\textbf{A:} \textit{Maylang sa permiro} (It was only good at the start). After about 1 month of implementation, citizens are now starting to segregate again.

\vspace{0.3cm}

\noindent \textbf{Q: In your opinion, what is the biggest challenge regarding resident behavior?} \\
\textbf{A:} They are aware but non-compliant. They need constant reminders and encouragement.

\vspace{0.3cm}

\noindent \textbf{Q: Does a resident's education level or income seem to affect their willingness to segregate?} \\
\textbf{A:} Education does not matter. Sometimes, the educated residents are actually the worst offenders.

\vspace{0.5cm}

%---------------------------------------------------------
%   THEME 4: CHALLENGES AND BARRIERS
%---------------------------------------------------------
\noindent \textbf{\large Part 4: Challenges and Barriers}
\vspace{0.2cm}

\noindent \textbf{Q: What specific challenges do low-income households face in complying with segregation?} \\
\textbf{A:} Low-income households often have no money to buy segregation bins. Even a sack (\textit{sako}) is considered expensive for them.

\vspace{0.3cm}

\noindent \textbf{Q: Does the barangay provide any assistance, like free sacks, to these households?} \\
\textbf{A:} We have issued sacks to low-income households from the barangay, but it is not constant due to budget constraints.

\vspace{0.3cm}

\noindent \textbf{Q: Have you received reports of collectors accepting extra payments to bypass the ``no segregation'' rule?} \\
\textbf{A:} Yes, there are instances where money added for non-segregated waste gets accepted.

\vspace{0.3cm}

\noindent \textbf{Q: What other challenges do you face in managing the barangay's solid waste?} \\
\textbf{A:} Security of facilities is an issue; at night, the MRF (Materials Recovery Facility) has been desecrated/vandalized.

\vspace{0.5cm}

%---------------------------------------------------------
%   THEME 5: LOGISTICS, BUDGET, AND INCENTIVES
%---------------------------------------------------------
\noindent \textbf{\large Part 5: Logistics, Budget, and Incentives}
\vspace{0.2cm}

\noindent \textbf{Q: What is the garbage collection schedule for the LGU truck?} \\
\textbf{A:} The LGU collects every Tuesday morning.

\vspace{0.3cm}

\noindent \textbf{Q: Does the barangay have its own vehicle for collection?} \\
\textbf{A:} We have a MultiCab but it is currently damaged. We have low maintenance capability for garbage collection vehicles. There are plans to acquire a new one.

\vspace{0.3cm}

\noindent \textbf{Q: What is your total budget for the solid waste management program for 2025?} \\
\textbf{A:} Our budget for 2025 is 30,000 Pesos.

\vspace{0.3cm}

\noindent \textbf{Q: Have you ever used incentives to encourage segregation?} \\
\textbf{A:} Yes, back in 2020, we had an SK (Sangguniang Kabataan) initiative where residents could exchange 3 eco-bricks for 1 kilo of rice, or 1 bottle for oil.

\vspace{0.3cm}

\noindent \textbf{Q: Are there plans to reintroduce an incentive program?} \\
\textbf{A:} Yes, we plan to bring back eco-brick incentives with a budget of 20,000 Pesos.

\vspace{0.3cm}

\noindent \textbf{Q: Besides collection, what other activities does the barangay conduct?} \\
\textbf{A:} We conduct ``Pulot Basura'' (waste picking), weekly \textit{pahina} (community cleanup), and monthly road clearing operations.