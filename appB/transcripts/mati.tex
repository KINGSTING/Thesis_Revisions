%---------------------------------------------------------
%   Barangay Mati
%---------------------------------------------------------
\section{Barangay Mati}
\label{appendix:mati_interview}

\noindent
\textbf{Resource Person:} Hon. Joel D. Bucol \\
\textbf{Position:} Barangay Chairman \\

\vspace{0.5cm}


\noindent \textbf{\large Part 1: The ``Config'' Variables (Demographics \& Baseline)}
\vspace{0.2cm}

\noindent \textbf{Q: For our simulation accuracy, what is the most recent count of households in your barangay?} \\
\textbf{A:} We have a total of 165 households.

\vspace{0.3cm}

\noindent \textbf{Q: If you had to estimate, what percentage of your residents are Low Income, Middle Income, and High Income?} \\
\textbf{A:} Most are farmers. I estimate 90\% Low Income, 5\% Middle, and 5\% High.

\vspace{0.3cm}

\noindent \textbf{Q: Honestly, right now, out of 10 households, how many are already strictly segregating their waste?} \\
\textbf{A:} We are newly adapting, but compliance is around 70\%.

\vspace{0.5cm}

\noindent \textbf{\large Part 2: The ``Budget \& Policy'' Variables (Resources)}
\vspace{0.2cm}

\noindent \textbf{Q: For Solid Waste Management (MOOE) specifically, how much did you roughly spend last year?} \\
\textbf{A:} Our environmental budget is roughly 80,000 Pesos (Total budget approx. 100,000 Pesos).

\vspace{0.3cm}

\noindent \textbf{Q: How much manpower do you have for this program?} \\
\textbf{A:} We have 1 Committee on Environment, 1 Sanitary Inspector, 1 Purok Leader, and 10 Barangay Tanods who act as collectors.

\vspace{0.3cm}

\noindent \textbf{Q: Regarding enforcement costs: Do you impose fines or penalties?} \\
\textbf{A:} We do not really enforce strict penalties yet as we are still adapting, but we do conduct inspections.

\vspace{0.3cm}

\noindent \textbf{Q: Regarding incentives: Do you have a rewards program?} \\
\textbf{A:} Yes, ``Basura mo, Bigas ko''. 1.5 Eco-bricks can be exchanged for 1 kilo of rice. We also have Cash for Work for surrenderers.

\vspace{0.5cm}

\noindent \textbf{\large Part 3: The ``Behavioral'' Variables (Household Logic)}
\vspace{0.2cm}

\noindent \textbf{Q: What is the ``Hassle Factor''? Do residents find it hard to segregate?} \\
\textbf{A:} Residents are aware, so the effort is low. However, segregation is noticeably easier for those living near the highway.

\vspace{0.3cm}

\noindent \textbf{Q: How strong are the Social Norms here? Do neighbors influence each other?} \\
\textbf{A:} Yes, reporting of illegal dumping is high because people are aware of potential punitive measures.

\vspace{0.3cm}

\noindent \textbf{Q: Have you noticed any ``Decay'' in attitude? Do they go back to old habits?} \\
\textbf{A:} Yes, there is high decay. There is an attitude problem where people slide back to old habits.

\vspace{0.5cm}

\noindent \textbf{\large Part 4: System Dynamics (Logistics \& Rules)}
\vspace{0.2cm}

\noindent \textbf{Q: How sensitive is the political situation to strict enforcement?} \\
\textbf{A:} It is high (Score: 8/10).

\vspace{0.3cm}

\noindent \textbf{Q: If you had extra budget, where would you put it?} \\
\textbf{A:} We would buy our own garbage truck and hire additional collectors. Residents would collect and segregate better if the Barangay had its own truck and a large MRF facility.
