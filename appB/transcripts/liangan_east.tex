%---------------------------------------------------------
%   Barangay Liangan East
%---------------------------------------------------------
\section{Barangay Liangan East}
\label{appendix:liangan_interview}

\noindent
\textbf{Resource Persons:} \\
1. Hon. Rufo Palangan Lumacad (Barangay Captain) \\
2. Ms. Analyn J. Eltagon (Barangay Secretary) \\
\textbf{Location:} Brgy. Liangan East, Bacolod, Lanao del Norte

\vspace{0.5cm}

\noindent \textbf{\large Part 1: Demographics and Profile}
\vspace{0.2cm}

\noindent \textbf{Q: (To Secretary) What is the total population and number of households in Barangay Liangan East?} \\
\textbf{A:} We have a total population of 2,198 individuals and a total of 608 households.

\vspace{0.3cm}

\noindent \textbf{Q: What would you say is the average household size?} \\
\textbf{A:} The average household size is about 4 members.

\vspace{0.3cm}

\noindent \textbf{Q: How would you describe the general income level of the residents?} \\
\textbf{A:} The general income level is middle class.

\vspace{0.3cm}

\noindent \textbf{Q: How many barangay officials and BPAT members do you have?} \\
\textbf{A:} We have 22 Barangay Officials and 10 Barangay Peacekeeping Action Team (BPAT) members.

\vspace{0.5cm}

\noindent \textbf{\large Part 2: Policies and Information Campaigns}
\vspace{0.2cm}

\noindent \textbf{Q: What official policies does the barangay have for solid waste management?} \\
\textbf{A:} We follow the Executive Order and the Municipal Ordinance.

\vspace{0.3cm}

\noindent \textbf{Q: How do you communicate these policies to the residents?} \\
\textbf{A:} We use Information Communication Campaigns and the Barangay Assembly. We also hold meetings by Purok, led by the Captain and Kagawad.

\vspace{0.3cm}

\noindent \textbf{Q: What is your single most important rule regarding household garbage collection?} \\
\textbf{A:} ``No Segregation, No Collection'' policy.

\vspace{0.5cm}

\noindent \textbf{\large Part 3: Enforcement and Community Compliance}
\vspace{0.2cm}

\noindent \textbf{Q: Who is responsible for enforcing the ``No Segregation, No Collection'' policy on the ground?} \\
\textbf{A:} The BPAT and Barangay Officials are responsible for enforcement.

\vspace{0.3cm}

\noindent \textbf{Q: What is the process when a household is non-compliant? Does the barangay issue a ticket?} \\
\textbf{A:} The Barangay does not issue tickets directly. We only report the violators. The MENRO is the one handling the citation tickets. The penalty comes from a summon by the MENRO and LGU based on our reports.

\vspace{0.3cm}

\noindent \textbf{Q: What is the current compliance rate for waste segregation among the households?} \\
\textbf{A:} It is around 60-70\% compliance.

\vspace{0.3cm}

\noindent \textbf{Q: Have you noticed this compliance changing over time?} \\
\textbf{A:} \textit{Maylang sa permiro} (It was only good at the start). After about 1 month of implementation, citizens are now starting to segregate again.

\vspace{0.3cm}

\noindent \textbf{Q: In your opinion, what is the biggest challenge regarding resident behavior?} \\
\textbf{A:} They are aware but non-compliant. They need constant reminders and encouragement.

\vspace{0.3cm}

\noindent \textbf{Q: Does a resident's education level or income seem to affect their willingness to segregate?} \\
\textbf{A:} Education does not matter. Sometimes, the educated residents are actually the worst offenders.

\vspace{0.5cm}

\noindent \textbf{\large Part 4: Challenges and Barriers}
\vspace{0.2cm}

\noindent \textbf{Q: What specific challenges do low-income households face in complying with segregation?} \\
\textbf{A:} Low-income households often have no money to buy segregation bins. Even a sack (\textit{sako}) is considered expensive for them.

\vspace{0.3cm}

\noindent \textbf{Q: Does the barangay provide any assistance, like free sacks, to these households?} \\
\textbf{A:} We have issued sacks to low-income households from the barangay, but it is not constant due to budget constraints.

\vspace{0.3cm}

\noindent \textbf{Q: Have you received reports of collectors accepting extra payments to bypass the ``no segregation'' rule?} \\
\textbf{A:} Yes, there are instances where money added for non-segregated waste gets accepted.

\vspace{0.3cm}

\noindent \textbf{Q: What other challenges do you face in managing the barangay's solid waste?} \\
\textbf{A:} Security of facilities is an issue; at night, the MRF (Materials Recovery Facility) has been desecrated/vandalized.

\vspace{0.5cm}

\noindent \textbf{\large Part 5: Logistics, Budget, and Incentives}
\vspace{0.2cm}

\noindent \textbf{Q: What is the garbage collection schedule for the LGU truck?} \\
\textbf{A:} The LGU collects every Tuesday morning.

\vspace{0.3cm}

\noindent \textbf{Q: Does the barangay have its own vehicle for collection?} \\
\textbf{A:} We have a MultiCab but it is currently damaged. We have low maintenance capability for garbage collection vehicles. There are plans to acquire a new one.

\vspace{0.3cm}

\noindent \textbf{Q: What is your total budget for the solid waste management program for 2025?} \\
\textbf{A:} Our budget for 2025 is 30,000 Pesos.

\vspace{0.3cm}

\noindent \textbf{Q: Have you ever used incentives to encourage segregation?} \\
\textbf{A:} Yes, back in 2020, we had an SK (Sangguniang Kabataan) initiative where residents could exchange 3 eco-bricks for 1 kilo of rice, or 1 bottle for oil.

\vspace{0.3cm}

\noindent \textbf{Q: Are there plans to reintroduce an incentive program?} \\
\textbf{A:} Yes, we plan to bring back eco-brick incentives with a budget of 20,000 Pesos.

\vspace{0.3cm}

\noindent \textbf{Q: Besides collection, what other activities does the barangay conduct?} \\
\textbf{A:} We conduct ``Pulot Basura'' (waste picking), weekly \textit{pahina} (community cleanup), and monthly road clearing operations.
