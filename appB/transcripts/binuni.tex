\section{Barangay Binuni}
\label{appendix:binuni_interview}

\noindent
\textbf{Resource Person:} Mr. Elven O. Baleos \\
\textbf{Position:} Barangay Secretary \\
\textbf{Date of Interview:} [Insert Date]

\vspace{0.5cm}

\noindent \textbf{\large Part 1: The ``Config'' Variables (Demographics \& Baseline)}
\vspace{0.2cm}

\noindent \textbf{Q: For our simulation accuracy, what is the most recent count of households in your barangay?} \\
\textbf{A:} We have a total of 507 households.

\vspace{0.3cm}

\noindent \textbf{Q: If you had to estimate, what percentage of your residents are Low Income, Middle Income, and High Income?} \\
\textbf{A:} I estimate 50\% Low Income, 30\% Middle, and 20\% High.

\vspace{0.3cm}

\noindent \textbf{Q: Honestly, right now, out of 10 households, how many are already strictly segregating their waste?} \\
\textbf{A:} We have a very high compliance rate, around 95\%. Residents are well-practiced and disciplined.

\vspace{0.5cm}

\noindent \textbf{\large Part 2: The ``Budget \& Policy'' Variables (Resources)}
\vspace{0.2cm}

\noindent \textbf{Q: For Solid Waste Management (MOOE) specifically, how much did you roughly spend last year?} \\
\textbf{A:} For 2026, our allocated budget is 126,370 Pesos.

\vspace{0.3cm}

\noindent \textbf{Q: How much manpower do you have for this program?} \\
\textbf{A:} We have a plantilla garbage collector (7,000 Pesos/month) who uses a pushcart (``Kariton''). We also hire RHU personnel for inspections.

\vspace{0.3cm}

\noindent \textbf{Q: Regarding enforcement costs: Do you impose fines or penalties?} \\
\textbf{A:} Yes, we issue citation tickets and enforce ``No Segregation, No Collection.'' The collector maintains a logbook of violators.

\vspace{0.3cm}

\noindent \textbf{Q: Regarding incentives: Do you have a rewards program?} \\
\textbf{A:} Yes. We hold a ``Clean and Green'' contest twice a year. We also have an Eco-brick dividend system where 20,000 Pesos is distributed in December to compliant households; the most compliant get the biggest share.

\vspace{0.5cm}

\noindent \textbf{\large Part 3: The ``Behavioral'' Variables (Household Logic)}
\vspace{0.2cm}

\noindent \textbf{Q: What is the ``Hassle Factor''? Do residents find it hard to segregate?} \\
\textbf{A:} The effort is low. It is well-practiced, and we require compost pits.

\vspace{0.3cm}

\noindent \textbf{Q: How strong are the Social Norms here? Do neighbors influence each other?} \\
\textbf{A:} Social norms are high. The garbage collector provides feedback to those who do not segregate.

\vspace{0.3cm}

\noindent \textbf{Q: Have you noticed any ``Decay'' in attitude? Do they go back to old habits?} \\
\textbf{A:} No, decay is low.

\vspace{0.5cm}

\noindent \textbf{\large Part 4: System Dynamics (Logistics \& Rules)}
\vspace{0.2cm}

\noindent \textbf{Q: Where does the money from fines go?} \\
\textbf{A:} Fines go back to the Barangay Budget. It is considered income-generating.

\vspace{0.3cm}

\noindent \textbf{Q: How sensitive is the political situation to strict enforcement?} \\
\textbf{A:} Political backlash is low.

\vspace{0.3cm}

\noindent \textbf{Q: If you had extra budget, where would you put it?} \\
\textbf{A:} We plan to purchase a garbage truck using the Calamity Fund (targeted for 2027).