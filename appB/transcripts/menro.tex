%---------------------------------------------------------
%   MENRO Head
%---------------------------------------------------------
\section{Municipal Environment and Natural Resources Officer (MENRO)}
\label{appendix:menro_interview}

\noindent
\textbf{Resource Person:} Engr. Archer M. Zamora \\
\textbf{Position:} MENRO Head, LGU-Bacolod \\

\vspace{0.5cm}

\noindent \textbf{\large Policy and Roles}
\vspace{0.2cm}

\noindent \textbf{Q: What are the primary laws and ordinances guiding your solid waste management (SWM) implementation?} \\
\textbf{A:} We are focused on implementing Republic Act 9003 (The Ecological Solid Waste Management Act). This includes enforcing local ordinances for the segregation of waste and the prohibition of single-use plastics.

\vspace{0.3cm}

\noindent \textbf{Q: How are SWM responsibilities divided between the LGU and the Barangays?} \\
\textbf{A:} The Barangay is the frontline of implementation. Their primary role is the collection of garbage at their level and the operation of their local Materials Recovery Facility (MRF). The LGU (Municipal) level is then responsible for collecting the waste from the Barangays, but we only collect waste that has already been properly segregated.

\vspace{0.5cm}

\noindent \textbf{\large Incentives and Penalties (The ``Carrot and Stick'')}
\vspace{0.2cm}

\noindent \textbf{Q: What incentives, or ``carrot'' approaches, do you use to encourage Barangays and households to comply?} \\
\textbf{A:} We use several ``carrot-style'' incentives. This includes:
\begin{itemize}
    \item Recognition for the best-implementing Barangays.
    \item Programs like Ecobrick exchange for goods.
    \item Before the pandemic, we held competitions, but this was stopped due to budget constraints.
\end{itemize}

\noindent \textbf{Q: What penalties, or ``stick'' approaches, are used for enforcement?} \\
\textbf{A:} For enforcement, the MENRO has an inspection team. We issue citation tickets for penalties to violators. We also utilize CSU (Civil Security Unit) or ``Eco-warrior'' enforcers to monitor compliance.

\vspace{0.5cm}

\noindent \textbf{\large Budget and Manpower}
\vspace{0.2cm}

\noindent \textbf{Q: What is the budget for your SWM programs?} \\
\textbf{A:} Our program budget is approximately 1.5 million pesos. This budget has to cover collection, biodiversity projects, and all Solid Waste Management activities.

\vspace{0.3cm}

\noindent \textbf{Q: What are the main challenges you face with enforcement and resources?} \\
\textbf{A:} Our main challenges are budget and manpower.
\begin{itemize}
    \item \textbf{Manpower:} We have a significant lack of staff for enforcement.
    \item \textbf{Budget:} We cannot employ more enforcers because of budget constraints. Frankly, the budget is \textit{kulang} (insufficient). Many of our plans are on a ``wishlist'' because of these limited funds.
\end{itemize}

\vspace{0.5cm}

\noindent \textbf{\large Awareness and Behavioral Challenges}
\vspace{0.2cm}

\noindent \textbf{Q: How do you handle Information, Education, and Communication (IEC) campaigns?} \\
\textbf{A:} We run continuous sanitation and IEC campaigns. Our most effective tool is the local radio station (101.3 Grace Covenant FM), and we consistently provide reminders during every assembly. This requires a specific budget for radio advertising.

\vspace{0.3cm}

\noindent \textbf{Q: What is the biggest obstacle to successful waste segregation?} \\
\textbf{A:} The biggest obstacles are behavioral and cultural. We produce waste every day. Even if we have a complete ordinance, it will not be successful if we don't get cooperation from the people. The main problems are social norms, acceptance, and behavioral constraints.

\vspace{0.3cm}

\noindent \textbf{Q: What is the current rate of segregation at the source?} \\
\textbf{A:} We estimate the segregation rate at the household source is only about 10\%. However, establishments (businesses) are generally compliant and do segregate their waste.

\vspace{0.3cm}

\noindent \textbf{Q: If compliance is low, why not strictly penalize all non-compliant households?} \\
\textbf{A:} The problem is acceptance. If we were to be extremely strict right now, all households would be penalized, which isn't feasible. We must balance enforcement with continuous awareness.

\vspace{0.5cm}

\noindent \textbf{\large Accountability and Logistics}
\vspace{0.2cm}

\noindent \textbf{Q: How are Barangay officials held accountable for implementing SWM?} \\
\textbf{A:} Accountability is handled in several ways:
\begin{itemize}
    \item \textbf{Monitoring:} The DENR (Department of Environment and Natural Resources) monitors compliance.
    \item \textbf{Council Meetings:} We hold mandatory quarterly meetings with the Solid Waste Management Council (SWMC), which includes Barangay Officials and the LGU.
    \item \textbf{Sanctions:} As provided by law, Barangay officials may be suspended for failure to comply. However, we prefer to focus on awareness and reminders because we view them as partners.
\end{itemize}

\vspace{0.3cm}

\noindent \textbf{Q: What is the long-term strategy for improving these numbers?} \\
\textbf{A:} We are following our 10-year Solid Waste Management plan, which focuses on consistent reminders and awareness campaigns.

\vspace{0.3cm}

\noindent \textbf{Q: How do you manage collection for inland Barangays that are hard to reach?} \\
\textbf{A:} The inland Barangays have their own MRF and segregation facilities. Collection is a major issue due to accessibility—they are \textit{layo na kaayo} (very far). They often have to use their own initiative, such as using a multipurpose vehicle, to deliver their segregated garbage to the collection points.

\vspace{0.3cm}

\noindent \textbf{Q: Which specific Barangays does the LGU collect segregated waste from?} \\
\textbf{A:} The LGU currently collects from 7 Barangays, these are:
\begin{enumerate}
    \item Liangan East
    \item Esperanza
    \item Poblacion
    \item Binuni
    \item Demologan
    \item Mati
    \item Babalaya
\end{enumerate}