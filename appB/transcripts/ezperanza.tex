\section{Barangay Ezperanza}
\label{appendix:ezperanza_interview}

\noindent
\textbf{Resource Person:} Ms. Leslie H. Lagudas \\
\textbf{Position:} Barangay Treasurer \\

\vspace{0.5cm}

\noindent \textbf{\large Part 1: The ``Config'' Variables (Demographics \& Baseline)}
\vspace{0.2cm}

\noindent \textbf{Q: For our simulation accuracy, what is the most recent count of households in your barangay?} \\
\textbf{A:} We have a total of 574 households.
\vspace{0.3cm}

\noindent \textbf{Q: If you had to estimate, what percentage of your residents are Low Income, Middle Income, and High Income?} \\
\textbf{A:} I estimate 20\% Low Income, 50\% Middle, and 30\% High.
\vspace{0.3cm}

\noindent \textbf{Q: Honestly, right now, out of 10 households, how many are already strictly segregating their waste?} \\
\textbf{A:} We have around 20\% compliance.
Residents typically use sacks or large plastic bags (``Laton ipasako'').

\vspace{0.5cm}

\noindent \textbf{\large Part 2: The ``Budget \& Policy'' Variables (Resources)}
\vspace{0.2cm}

\noindent \textbf{Q: For Solid Waste Management (MOOE) specifically, how much did you roughly spend last year?} \\
\textbf{A:} Our budget is 90,000 Pesos.
\vspace{0.3cm}

\noindent \textbf{Q: How much manpower do you have for this program?} \\
\textbf{A:} We have 3 assigned street garbage collectors.
We also have 2 ``Elf'' trucks (volunteer) to dump at the Eco Park.

\vspace{0.3cm}

\noindent \textbf{Q: Regarding enforcement costs: Do you impose fines or penalties?} \\
\textbf{A:} Currently, there is no ticketing or penalty.
However, we plan to enforce it this year. We have ``Eco-police'' in Purok 1.

\vspace{0.3cm}

\noindent \textbf{Q: Regarding incentives: Do you have a rewards program?} \\
\textbf{A:} We provide incentives for the collectors.
However, we say ``No'' to material incentives for households (like garbage-to-food programs) because we worry households will rely on the incentive rather than discipline.

\vspace{0.5cm}

\noindent \textbf{\large Part 3: The ``Behavioral'' Variables (Household Logic)}
\vspace{0.2cm}

\noindent \textbf{Q: What is the ``Hassle Factor''? Do residents find it hard to segregate?} \\
\textbf{A:} The effort is low, but the mental attitude/appreciation of the ordinance is lacking.
\vspace{0.3cm}

\noindent \textbf{Q: How strong are the Social Norms here? Do neighbors influence each other?} \\
\textbf{A:} Social norms are high.
However, illegal dumping still occurs on roadsides.

\vspace{0.3cm}

\noindent \textbf{Q: Have you noticed any ``Decay'' in attitude? Do they go back to old habits?} \\
\textbf{A:} Decay is low; we do ``Recoreda'' (announcements).
\vspace{0.5cm}

\noindent \textbf{\large Part 4: System Dynamics (Logistics \& Rules)}
\vspace{0.2cm}

\noindent \textbf{Q: Where does the money from fines go?} \\
\textbf{A:} Collected fines are accounted to barangay funds.

\vspace{0.3cm}

\noindent \textbf{Q: How sensitive is the political situation to strict enforcement?} \\
\textbf{A:} Political sensitivity is low.
Officials are willing to explain to the people (``Pagpasabot sa ilaha'').