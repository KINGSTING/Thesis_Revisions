%---------------------------------------------------------
%   Barangay Babalaya
%---------------------------------------------------------
\section{Barangay Babalaya}
\label{appendix:babalaya_interview}

\noindent
\textbf{Resource Person:} Hon. Antonette A. Geman \\
\textbf{Position:} Barangay Kagawad, Committee on SWM \\

\vspace{0.5cm}

\noindent \textbf{\large Part 1: The ``Config'' Variables (Demographics \& Baseline)}
\vspace{0.2cm}

\noindent \textbf{Q: For our simulation accuracy, what is the most recent count of households in your barangay?} \\
\textbf{A:} We have a total of 171 households.

\vspace{0.3cm}

\noindent \textbf{Q: If you had to estimate, what percentage of your residents are Low Income, Middle Income, and High Income?} \\
\textbf{A:} It is roughly 80\% Low Income, 10\% Middle, and 10\% Upper. (Note: Respondent added that low-income households actively recycle rice sacks due to necessity).

\vspace{0.3cm}

\noindent \textbf{Q: Honestly, right now, out of 10 households, how many are already strictly segregating their waste?} \\
\textbf{A:} I would say 100\% are compliant.

\vspace{0.5cm}

\noindent \textbf{\large Part 2: The ``Budget \& Policy'' Variables (Resources)}
\vspace{0.2cm}

\noindent \textbf{Q: For Solid Waste Management (MOOE) specifically, how much did you roughly spend last year?} \\
\textbf{A:} Our current budget is 15,000.00 Pesos.

\vspace{0.3cm}

\noindent \textbf{Q: How much manpower do you have for this program?} \\
\textbf{A:} We have 7 personnel in total, composed of the Committee on SWM and the Barangay Health Workers (BHW).

\vspace{0.3cm}

\noindent \textbf{Q: Regarding enforcement costs: Do you impose fines or penalties?} \\
\textbf{A:} No penalty, we just remind them (``Just remind''). We have a designated person (Ganiron) who checks on households.

\vspace{0.3cm}

\noindent \textbf{Q: Regarding incentives: Do you have a rewards program?} \\
\textbf{A:} Yes. We have an Eco-bricks for Rice (``bugas'') program. Households need 25 eco-bricks to claim the incentive.

\vspace{0.5cm}

\noindent \textbf{\large Part 3: The ``Behavioral'' Variables (Household Logic)}
\vspace{0.2cm}

\noindent \textbf{Q: What is the ``Hassle Factor''? Do residents find it hard to segregate?} \\
\textbf{A:} No, they do not find it difficult. In their words, ``Dili ra effort mag segre'' (It's not much effort to segregate).

\vspace{0.3cm}

\noindent \textbf{Q: How strong are the Social Norms here? Do neighbors influence each other?} \\
\textbf{A:} Yes, there is high social pressure (``High social IEC from social norms''). The community norm effectively acts as the information campaign.

\vspace{0.3cm}

\noindent \textbf{Q: Have you noticed any ``Decay'' in attitude? Do they go back to old habits?} \\
\textbf{A:} There is Low Decay. Some residents express fatigue (``Sumhan na''/fed up), but the system still works, and the attitude towards segregation remains functional.

\vspace{0.5cm}

\noindent \textbf{\large Part 4: System Dynamics (Logistics \& Rules)}
\vspace{0.2cm}

\noindent \textbf{Q: How often does the Municipal Truck collect waste?} \\
\textbf{A:} The Municipal truck comes very seldom.

\vspace{0.3cm}

\noindent \textbf{Q: If the truck is seldom, how do residents manage their residuals?} \\
\textbf{A:} Most residents use compost pits here within the barangay.