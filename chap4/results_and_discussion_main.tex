\chapter{Results \& Discussion}
This chapter presents the empirical parameterization of the simulation environment, the calibration of the Agent-Based Model (ABM) against real-world audits, and the comparative analysis of policy scenarios generated by the Deep Reinforcement Learning (DRL) agent.

% =========================================
% SECTION 4.1: INPUTS AND PARAMETERS
% =========================================
\section{Analysis of System Inputs and Parameters}
The validity of an Agent-Based Model (ABM) relies heavily on the empirical grounding of its initialization parameters. Before training the Deep Reinforcement Learning (DRL) agent, the real-world data gathered from the Municipality of Bacolod—comprising legislative documents and key informant interviews—were analyzed to define the state space, constraints, and reward structures of the simulation environment. This section details how qualitative field data was translated into quantitative simulation parameters.

\subsection{Legislative Constraints and Financial Bounds}
Based on the simulation parameters defined in the barangay configuration, the interview transcripts referenced in Appendix B, and the financial data provided by the Municipal Environment and Natural Resources Office (MENRO), this section details the legislative and financial constraints that define the solution space for the Agent-Based Model. The simulation environment is strictly bounded by the fiscal realities of the Municipality of Bacolod. Unlike theoretical models where resources may be infinite, this study incorporates hard constraints derived from Appropriation Ordinance No. 2024-01, ensuring that the Deep Reinforcement Learning (DRL) agent operates within the actual legal and financial framework of the Local Government Unit (LGU).

At the macro-environmental level, the MENRO acts as the primary resource allocator. The simulation enforces a strict Annual Budget Cap of \textpeso 1,500,000, a figure calibrated directly from the MENRO interview detailed in Appendix B.1. This creates a quarterly resource constraint of \textpeso 375,000. Under the ``Status Quo'' policy, these funds are typically distributed based on population density; however, the simulation reveals that a purely demographic distribution dilutes the impact in high-density areas while over-funding low-density areas. This confirms the ``institutional failure'' described by Iba{\~n}ez and Jr \cite{Ibanez2022} and Santos \cite{Santos2025}, who argue that systemic funding deficits fundamentally weaken the implementation of RA 9003 at the municipal level.

Furthermore, operational constraints are dictated by the regional economic reality. With a Daily Minimum Wage of \textpeso 400 based on Region X rates, the cost of human enforcement is significantly high. A single enforcer patrolling for a quarter consumes approximately \textpeso 24,000 to \textpeso 30,000. This financial reality creates a ``Personnel Trap'' where the budget is consumed entirely by salaries, leaving zero room for Information, Education, and Communication (IEC) campaigns. This validates the observation by Nishimura \cite{Nishimura2022} regarding the operational burden on LGUs, where the high cost of technical and human resources creates a bottleneck that prevents holistic program implementation.

\begin{table}[htbp]
    \centering
    \caption{Barangay Demographic and Financial Initialization Parameters}
    \label{tab:barangay_demographics}
    \small 
    \begin{tabular}{l r r c c c c}
        \toprule
        & & {Annual} & {Initial} & \multicolumn{3}{c}{{Income Profile (\%)}} \\
        \cmidrule(lr){5-7}
        {Barangay} & {Households} & {Budget (\textpeso)} & {Compliance} & {Low} & {Mid} & {High} \\
        \midrule
        Brgy Poblacion    & 1,534 & 200,000 & 2\%  & 70 & 25 & 5 \\
        Brgy Liangan East & 608   & 30,000  & 14\% & 40 & 40 & 20 \\
        Brgy Ezperanza    & 574   & 90,000  & 14\% & 20 & 50 & 30 \\
        Brgy Binuni       & 507   & 126,370 & 15\% & 50 & 30 & 20 \\
        Brgy Demologan    & 463   & 21,000  & 11\% & 80 & 15 & 5 \\
        Brgy Babalaya     & 171   & 15,000  & 14\% & 80 & 10 & 10 \\
        Brgy Mati         & 165   & 80,000  & 11\% & 90 & 5  & 5 \\
        \bottomrule
    \end{tabular}
    \vspace{1ex}
    
    \raggedright
    \footnotesize
    \textit{Note: Income profiles derived from Appendix B interviews. Low Income agents have higher price sensitivity ($\gamma$) to fines.}
\end{table}

The heterogeneity of the seven barangays creates a complex optimization landscape where each operates under unique financial bounds and legislative priorities. Barangay Poblacion represents an urban resource trap. Despite having the highest local budget of \textpeso 200,000, it suffers from the lowest per-capita spending power due to its massive population of 1,534 households. Defined as the ``Police State'' profile in Appendix B.8, Poblacion is legislatively bound to allocate 60\% of its budget to enforcement because the high density of commerce and residence necessitates a heavy presence of Barangay Tanods to maintain basic order. The simulation demonstrates that despite the high absolute budget, this 60\% enforcement lock leaves insufficient funds for IEC or incentives, resulting in a rigid compliance ceiling of approximately 2\%. This aligns with findings by Collado et al. \cite{Collado2024}, who noted that while strict enforcement policies like ``No Segregation, No Collection'' exist, they are often unsustainable without complementary behavioral interventions.

Conversely, Barangay Binuni represents the simulation’s ``Best Case Scenario'' or wealthy anomaly. With a budget of \textpeso 126,370 for only 507 households, it enjoys a high per-capita budget. This fiscal surplus allows Binuni to adopt a balanced ``Rich \& Capable'' profile, as noted in Appendix B.3, splitting funds evenly between enforcement and incentives. This financial flexibility allows Binuni to overcome high behavioral friction, making it the consistent top performer in the Status Quo analysis.

Barangay Liangan East faces a critical mismatch that highlights the risks of incentive-heavy programs without sufficient funding. It attempts an ambitious program with a meager budget of \textpeso 30,000 for 608 households. Modeled after the ``Eco-brick'' project mentioned in Appendix B.6, Liangan East is legislatively biased toward incentives at 65\%, leaving only 25\% for enforcement. This creates a fragile system where the simulation shows that without LGU augmentation, the \textpeso 30,000 is quickly exhausted by incentive redemptions, leading to rapid program collapse. This validates the ``False Spike'' behavioral profile, where compliance rises briefly but crashes when funds run out. This phenomenon mirrors the conclusions of Camarillo and Bellotindos \cite{Camarillo2021}, who emphasized that while incentive programs are attractive, their long-term effectiveness is often volatile and dependent on continuous financial support.

In contrast, Barangay Ezperanza represents the middle-class standard. With \textpeso 90,000 for 574 households, it serves as the median control group. Following a ``Consistent'' profile from Appendix B.5, it maintains a 50\% enforcement and 30\% incentive split, allowing it to maintain a baseline compliance of roughly 10\% without external intervention.

Barangay Mati acts as a small-scale surplus outlier with a high budget of \textpeso 80,000 for a tiny population of 165 households. Being 90\% agricultural as described in Appendix B.7, the legislative focus is on incentives rather than strict policing. The high budget-to-household ratio means Mati is highly responsive to financial injections; however, the simulation assigns a high decay rate to this profile, indicating that while farmers participate for rewards, the habit formation is volatile.

Finally, Barangays Babalaya and Demologan represent the poverty trap of the simulation, operating with budgets of \textpeso 15,000 and \textpeso 21,000 respectively. They are bound by the ``Survivor'' profile detailed in Appendices B.2 and B.4. With such limited funds, 85\% to 90\% of the budget is locked into enforcement, not as a strategic choice but a survival necessity. \textpeso 15,000 is barely enough to pay the honoraria of two Tanods for a quarter. Consequently, their IEC and incentive allocation is effectively zero. The simulation confirms that under the Status Quo, these barangays are mathematically incapable of improving compliance without external LGU subsidy. This provides computational evidence for the argument by Espino et al. \cite{Espino2025} that non-compliance in under-resourced areas is often driven by structural limitations rather than mere negligence.

\subsection{Heterogeneity of Barangay Operations and Behavior}
A critical finding from the data gathering phase (Appendix B) is that the Municipality of Bacolod cannot be modeled as a monolithic entity. The simulation environment is characterized by profound heterogeneity across its seven constituent barangays, creating a non-convex optimization landscape for the Deep Reinforcement Learning (DRL) agent. A ``one-size-fits-all'' policy is mathematically destined to fail because the responsiveness of agents varies drastically depending on their local context. This supports the core argument of Zheng et al. \cite{Zheng2022}, which posits that policy effectiveness is non-uniform and highly contingent on household heterogeneity, specifically income and education levels.

The economic landscape of the municipality is sharply divided between agrarian subsistence zones and emerging commercial centers, a disparity that fundamentally alters the marginal utility of money within the simulation’s utility function. According to the income profiles calibrated from Appendix B, Barangays Babalaya and Demologan represent the municipality's economic floor, with 80\% of their households classified as low-income earners. In the Agent-Based Model, this high concentration of poverty assigns a higher gamma value to these agents, meaning that monetary fines are disproportionately punitive, yet the opportunity cost of time (compliance effort) is critically high due to subsistence labor. This simulation outcome reinforces Badua \cite{Badua2022}, who warns that financial penalties disproportionately affect low-income groups, rendering uniform punitive policies regressive and potentially unjust.

In stark contrast, Barangay Binuni exhibits a ``Wealthy Anomaly'' profile, comprising a mix of 50\% low-income, 30\% middle-income, and 20\% high-income households. This economic buffer reduces the friction of compliance, as wealthier households possess the resources to manage waste segregation without sacrificing essential labor time. Barangay Ezperanza serves as the emerging middle-class standard, with a balanced distribution that creates a predictable response to economic incentives. This heterogeneity implies that a uniform fine of \textpeso 500 is mathematically devastating for a household in Mati (90\% low-income/farmers) but merely a nuisance for a household in Binuni, forcing the DRL agent to weigh the equity of enforcement across these disparate economic zones. This disparity highlights the need for the ``robust optimization'' approach advocated by Gentile et al. \cite{Gentile2022}, where policy must be designed to withstand parameter uncertainty across diverse populations.

\begin{table}[htbp]
    \centering
    \caption{Calibrated Behavioral Weights and Legislative Allocation Strategies}
    \label{tab:behavioral_parameters}
    \resizebox{\textwidth}{!}{ 
    \begin{tabular}{l c c c c c c c c}
        \toprule
        & \multicolumn{4}{c}{{Agent Behavior Parameters}} & & \multicolumn{3}{c}{{Budget Allocation (\%)}} \\
        \cmidrule(lr){2-5} \cmidrule(lr){7-9}
        {Barangay} & {Attitude ($w_a$)} & {Norms ($w_{sn}$)} & {Effort ($C_e$)} & {Decay ($\gamma$)} & & {Enf} & {Inc} & {IEC} \\
        \midrule
        Binuni       & 0.75 & 0.80 & 0.64 & 0.005 & & 40 & 40 & 20 \\
        Ezperanza    & 0.40 & 0.70 & 0.54 & 0.015 & & 50 & 30 & 20 \\
        Babalaya     & 0.80 & 0.90 & 0.62 & 0.030 & & 90 & 5  & 5 \\
        Liangan East & 0.65 & 0.60 & 0.58 & 0.050 & & 25 & 65 & 10 \\
        Poblacion    & 0.55 & 0.20 & 0.70 & 0.100 & & 60 & 20 & 20 \\
        Demologan    & 0.70 & 0.60 & 0.62 & 0.080 & & 85 & 10 & 5 \\
        Mati         & 0.60 & 0.50 & 0.60 & 0.100 & & 30 & 50 & 20 \\
        \bottomrule
    \end{tabular}
    }
    \vspace{1ex}
    
    \raggedright
    \footnotesize
    \textit{Note: Agent behavior parameters correspond to Theory of Planned Behavior (TPB) weights: $w_a$ (Attitude), $w_{sn}$ (Social Norms), $C_e$ (Perceived Cost of Effort), and $\gamma$ (Compliance Decay Rate). Budget allocation percentages reflect the legislative constraints under the Status Quo scenario.}
\end{table}

Beyond economics, the distinct psychosocial profiles of each barangay determine how they respond to policy interventions. The simulation encodes these differences through TPB weights—Attitude, Social Norms, and Perceived Behavioral Control—and a decay factor representing habit retention. Barangay Poblacion presents the most significant behavioral challenge; despite its size, it exhibits an ``Urban Isolation'' profile with the lowest Social Norm weight (0.20) and the highest Cost of Effort (0.75). This indicates that social pressure from neighbors is ineffective in the high-density commercial center, and the friction to comply is maximal. This finding is consistent with Meng et al. \cite{Meng2018}, who demonstrated that recycling behavior is heavily dependent on neighborhood-level interactions; where such social cohesion is absent (as in high-density urban settings), Subjective Norms fail to drive compliance.

Conversely, Barangay Babalaya demonstrates the ``Tight-Knit Community'' paradox. It possesses the highest Social Norm weight (0.90) and Attitude (0.80), suggesting a population that deeply desires to comply due to strong community cohesion. However, this is offset by the rigid financial constraints mentioned previously. Furthermore, the model accounts for the persistence of education through variable decay rates. Barangay Mati, representing the agricultural sector, has the highest decay rate (0.10), implying that IEC campaigns are quickly forgotten as residents revert to traditional practices. This necessitates the ``multi-pronged interventions'' recommended by Trushna et al. \cite{Trushna2024}, validating that information campaigns must be continuous rather than one-off events to combat behavioral decay.

% =========================================
% SECTION 4.2: CALIBRATION
% =========================================
\section{Model Calibration and Validation}
Model calibration serves as a critical validation step to ensure that the simulation accurately reproduces the current real-world state of the Municipality of Bacolod prior to the introduction of optimization policies. This process necessitates reconciling the discrepancies observed between micro-level data gathered through barangay interviews and the macro-level reality established by the Municipal Environment and Natural Resources Office audit.

\subsection{Addressing Reporting Bias in Input Data}
A significant challenge identified during the initialization phase was the marked divergence between self-reported compliance rates and the observed municipal average. Aggregating the compliance rates claimed by barangay officials yielded a calculated municipal average ranging from approximately 40\% to 50\%, with specific localities such as Binuni and Demologan reporting rates as high as 95\% and 60\%, respectively. However, the municipal-wide audit conducted by the MENRO indicated an actual effective segregation rate of approximately 10\%. This discrepancy quantitatively captures the critical ``Act vs. Reality'' gap described by Yazawa et al. \cite{Yazawa2025}, where reported compliance at the barangay level frequently deviates significantly from the actual ground-level implementation of RA 9003.

This discrepancy suggests the presence of social desirability bias in the interview data, where local officials may have overstated compliance to present a favorable political image. It also corroborates the findings of Abordo and Dalugdog \cite{Abordo2025}, who observed that high levels of reported awareness or ``policy knowledge'' often fail to translate into consistent, proper segregation practice. Consequently, the simulation utilizes the MENRO audit of 10\% as the ground truth target for calibration.

\subsection{Calibration of Initial Compliance States}
A critical divergence was observed between the compliance rates self-reported by barangay officials during Key Informant Interviews (Appendix B) and the objective audit data provided by the Municipal Environment and Natural Resources Office (MENRO). While aggregate interview data suggested a municipal compliance rate exceeding 50\%, the MENRO’s waste characterization study confirmed a functional segregation rate of only 10\% to 15\%.

To ensure the Agent-Based Model reflects the true ``Status Quo,'' the initialization parameters in \texttt{barangay\_config.py} were calibrated to match the MENRO ground truth rather than the optimistic self-reports. This decision rests on three specific analytical pivots:

\begin{table}[htbp]
    \centering
    \caption{Comparison of Acquired Compliance Data vs. Calibrated Simulation Initialization}
    \label{tab:calibration_data}
    \small
    \begin{tabular}{l c c l}
        \toprule
        {Barangay} & {Acquired Data} & {Calibrated $S_0$} & {Primary Calibration Factor} \\
        & \textit{(Self-Reported)} & \textit{(Simulation Start)} & \\
        \midrule
        Brgy Babalaya     & 100\% & 14\% & Adjusted for lack of funding/monitoring \\
        Brgy Binuni       & 95\%  & 15\% & Decoupled awareness from actual practice \\
        Brgy Mati         & 70\%  & 11\% & Adjusted for high habit decay rate \\
        Brgy Liangan East & 65\%  & 14\% & Removed temporary project-based spikes \\
        Brgy Demologan    & 60\%  & 11\% & Standardized to municipal baseline \\
        Brgy Ezperanza    & 20\%  & 14\% & Aligned with middle-class control group \\
        Brgy Poblacion    & 2\%   & 2\%  & {Anchor Point} (Aligned with Audit) \\
        \bottomrule
    \end{tabular}
    \vspace{1ex}
    
    \raggedright
    \footnotesize
    \textit{Note: "Acquired Data" refers to unverified compliance rates reported during Key Informant Interviews (Appendix B). "Calibrated $S_0$" refers to the initialized compliance state in the Agent-Based Model, tuned to match the aggregate municipal segregation rate of $\approx$12\% verified by MENRO.}
\end{table}

Barangay Binuni officials reported a near-perfect compliance rate of 95\% (Appendix B.3). However, initializing the simulation at this level resulted in immediate global compliance exceeding 40\%, rendering the RL agent’s optimization tasks trivial. The simulation logic posits that the reported 95\% reflects \textit{awareness} rather than \textit{practice}. This distinction is critical; as noted by \cite{Paigalan2025}, positive attitudes (Mean=3.17) often coexist with poor practices like open burning (Mean=2.90), creating an ``Intention-Action Gap'' that the model must explicitly correct for.

Barangay Liangan East reported 60\% compliance driven by an Eco-brick initiative. However, longitudinal analysis suggests that project-based compliance is transient. To model the ``Status Quo'' before the DRL agent's intervention, the system initializes Liangan East at 14\%. This reflects the baseline habituated behavior of the population when specific project funds are not actively being disbursed, preventing the simulation from starting at an artificial peak.

The only data point where self-reporting matched the MENRO audit was Barangay Poblacion, which reported a 2\% compliance rate. This convergence serves as the \textit{Anchor Point} for the simulation. Because the qualitative data and quantitative audit agreed on the lower bound, the simulation initializes Poblacion at 2\%. This validates the ``Urban Resource Trap'' theory, where high population density and anonymity dissolve social pressure, creating a distinct legislative challenge that the DRL agent must solve through enforcement rather than social norms. This low baseline confirms the assessment by Villanueva et al. \cite{Villanueva2021} that without strict monitoring mechanisms, the ``status quo'' in high-density areas defaults to non-compliance despite the existence of legal frameworks.

By calibrating the initial states to this conservative baseline (2\% to 15\%), the simulation stabilizes at a Global Average Compliance of $\approx$12.5\% at $t=0$. This provides a valid, non-inflated baseline against which the DRL agent's policy improvements can be accurately measured.

% =========================================
% SECTION 4.3: COMPARATIVE ANALYSIS
% =========================================
\section{Comparative Analysis of Policy Strategies}

To rigorously evaluate the efficacy of the proposed computational framework, the simulation was subjected to four distinct policy scenarios: (1) the Status Quo, characterized by an equal distribution of funds with a focus on Information, Education, and Communication (IEC); (2) a Pure Enforcement strategy, prioritizing punitive measures; (3) a Pure Incentives strategy, focusing solely on monetary rewards; and (4) the Heuristic-Guided PPO Algorithm, representing the adaptive, AI-driven approach. The results from these simulations unequivocally demonstrate that without the ``Sequential Saturation'' logic discovered by the artificial intelligence, no amount of static rebalancing between Education, Enforcement, or Incentives is sufficient to resolve the municipality-wide non-compliance problem.

\subsection{Baseline: The Status Quo (Equal Distribution - IEC Focused)}

The baseline scenario replicates the current administrative strategy employed by the Local Government Unit (LGU), wherein the limited municipal budget is distributed equally among all seven barangays, resulting in an allocation of approximately \textpeso 53,571 per quarter for each unit. This strategy heavily prioritizes Information, Education, and Communication (IEC) materials under the assumption that increasing awareness will drive behavioral change. The simulation results by Quarter 12 reveal a highly stratified outcome that is strictly determined by the initial demographic profile of each barangay. Small, socially cohesive areas such as Barangay Mati and Barangay Babalaya achieved 100\% compliance, driven not by the LGU's funding but by the inherent social momentum within their tight-knit populations.

\begin{figure}[htbp]
    \centering
    % Ensure the filename matches exactly (case-sensitive on Overleaf/Linux)
    \includegraphics[width=0.8\textwidth]{Baseline.jpg}
    \caption{Comparative compliance rates under the Status Quo strategy (Equal Distribution)}
    \label{fig:baseline_graph}
\end{figure}

However, this apparent success in smaller communities masks a catastrophic failure in the municipality's critical zones. Barangay Poblacion, the commercial center and largest generator of waste, stagnated at a negligible 0.33\% compliance rate. This bifurcation confirms the ``Urban Resource Trap'' hypothesis, where high population density and anonymity dissolve the social pressure that facilitates compliance in rural areas. The failure of the ``Status Quo'' demonstrates that an equal distribution strategy acts as a dilution mechanism; by spreading resources thinly across all areas, the LGU fails to generate the threshold intensity of enforcement or incentives required to break the behavioral resistance in stubborn, urbanized populations. This validates the assessment by \citet{Villanueva2021} that without strict, concentrated monitoring, high-density areas default to non-compliance despite the presence of legal frameworks.

\subsection{Alternative 1: Pure Enforcement Strategy}

In the second experimental scenario, the entire LGU allocation was shifted toward hiring Enforcement Agents (Tanods) while maintaining the principle of equal distribution. This strategy tested the ``deterrence hypothesis,'' which posits that increasing the probability of detection through stricter policing is the most effective driver of compliance. Contrary to this hypothesis, the Pure Enforcement strategy performed the worst overall among all tested scenarios. While it had a marginal impact, critical areas saw a collapse in participation: Barangay Poblacion remained stagnant at 0.00\%, and Barangay Liangan East regressed to a mere 1.97\% compliance.

\begin{figure}[htbp]
    \centering
    % Ensure the filename matches exactly (case-sensitive on Overleaf/Linux)
    \includegraphics[width=0.8\textwidth]{pure_enforcement.jpg}
    \caption{Compliance rates under the Pure Enforcement strategy (Equal Distribution)}
    \label{fig:pure_enforcement-graph}
\end{figure}

The data reveals a critical insight regarding the economics of coercion: diluted enforcement is effectively useless. Even with 100\% of the municipal funds allocated to hiring police, the equal distribution model meant that Poblacion's enforcement intensity never reached the critical threshold required to make non-compliance perceived as risky. The 608 households of Liangan East and the 1,534 households of Poblacion simply overwhelmed the limited number of enforcers that an equal-share budget could afford. Furthermore, the total absence of IEC and incentives eliminated any mechanism to improve the agents' Attitude or Perceived Behavioral Control, creating a purely punitive system that failed to sustain positive behavior. This limitation aligns with the findings of \citet{Badua2022}, who highlighted that strictly punitive measures are often ineffective and unsustainable if not supported by broader implementation strategies that address the root causes of non-compliance.

\subsection{Alternative 2: Pure Incentives Strategy}

The third scenario allocated all municipal funds to providing monetary rewards for compliant households, testing the efficacy of a purely positive reinforcement model. The results demonstrated moderate success in smaller demographics but ultimately failed in larger populations due to the mechanics of ``budget dilution.'' Barangay Binuni, with a moderate population, achieved a respectable 66.27\% compliance, and the smaller Barangay Babalaya reached 88.89\%. This success in lower-density areas confirms that economic rewards can effectively reduce the Cost of Effort when the per-capita reward value is sufficiently high.

\begin{figure}[htbp]
    \centering
    % Ensure the filename matches exactly (case-sensitive on Overleaf/Linux)
    \includegraphics[width=0.8\textwidth]{pure_incentives.jpg}
    \caption{Compliance rates under the Pure Incentives strategy (Equal Distribution)}
    \label{fig:pure_incentives_graph}
\end{figure}

However, the strategy failed decisively in the municipality's core. Barangay Poblacion stagnated at 0.00\% compliance, mirroring the failure of the Status Quo. Mathematically, incentives function by increasing the utility function of agents; however, in a large population like Poblacion (1,534 households), the fixed budget spread across so many potential recipients resulted in a per-capita reward that was negligible. Rational agents simply ignored the microscopic reward as it did not offset the effort required to segregate waste. This outcome validates the finding by \citet{Camarillo2021} that economic incentives require a saturation point to be effective; if the reward does not meet a minimum threshold of value, it fails to trigger the desired behavioral shift, proving that ``fair'' distribution often results in ineffective policy.

\subsection{Proposed: Heuristic-Guided Deep Reinforcement Learning Approach}

The Deep Reinforcement Learning (PPO) agent, operating under the ``Heuristic-Guided'' framework, completely outperformed all manual strategies by discovering a non-linear resource allocation logic termed ``Sequential Saturation.'' By Quarter 12, the AI-driven strategy achieved global success, with every single barangay crossing the 70\% ``graduation threshold.'' Most notably, Barangay Poblacion reached 74.51\% compliance—compared to near-zero in all other scenarios—while Barangay Liangan East and Barangay Demologan achieved 84.21\% and 97.19\% respectively.

\begin{figure}[htbp]
    \centering
    % Ensure the filename matches exactly (case-sensitive on Overleaf/Linux)
    \includegraphics[width=0.8\textwidth]{PPO.jpg}
    \caption{Compliance evolution under the Heuristic-Guided DRL strategy (Sequential Saturation)}
    \label{fig:PPO_graph}
\end{figure}

The superior performance of this algorithm is attributed to its abandonment of the ``Equal Distribution'' heuristic. The simulation logs reveal that the AI did not fight on all fronts simultaneously; instead, it concentrated approximately 90\% to 96\% of the total quarterly budget onto specific, single barangays for extended periods. For instance, the agent executed a dedicated ``siege'' on Poblacion from quarters 8 to 11, directing over \textpeso 320,000 per quarter exclusively to this target. This concentration of capital artificially lowered the cost of compliance and raised enforcement visibility to levels that the equal distribution strategy could never mathematically achieve.

This ``Sequential Saturation'' or ``King of the Hill'' logic validates the hypothesis of a social ``tipping point.'' The AI learned that by saturating one community until it reached a self-sustaining level of compliance (driven by social norms), it could then safely withdraw funds and move the ``siege'' to the next non-compliant barangay without causing a collapse in previous gains. This finding fundamentally challenges the prevailing administrative logic of equitable distribution, suggesting instead that effectiveness in resource-constrained environments requires the political will to prioritize sequential efficacy over simultaneous but ineffectual fairness. This conclusion provides a robust, data-driven roadmap for the LGU, proving that the binding constraint is not the total amount of money, but the strategy by which it is deployed.

% =========================================
% SECTION 4.4: MECHANISMS
% =========================================
\section{Mechanisms of Behavior Change and Sustainability}

The efficacy of the Deep Reinforcement Learning (DRL) strategy relies on its ability to exploit two underlying behavioral mechanisms uncovered during the simulation: the stabilizing force of ``Cultural Inertia'' and the opposing friction of ``Density-Dependent Resistance.'' Unlike the static policy regimes, the adaptive agent successfully identified these non-linear dynamics, leveraging social momentum to sustain compliance in rural areas while concentrating capital to overcome the structural inertia of urban centers.

\subsection{Cultural Inertia and the ``Graduation'' Effect}

A critical finding of this study is the pivotal role of Social Norms in sustaining compliance behavior even after the withdrawal of direct financial intervention. This phenomenon, observed in barangays such as Mati and Binuni, manifests as ``Cultural Inertia,'' where a community maintains high segregation rates despite the DRL agent cutting the local budget to near-zero ``Maintenance Mode'' levels. For instance, the simulation data reveals that Barangay Mati maintained 100\% compliance from Quarter 4 through Quarter 12, even as the agent reallocated the majority of its funding to other, more problematic areas. This persistence suggests that once a specific compliance threshold is crossed, the behavior becomes self-reinforcing, decoupling the action from the immediate presence of monetary rewards or enforcement risks.

This mechanism aligns with the theoretical framework established by \citet{Meng2018}, who demonstrated that household recycling behavior is heavily dependent on neighborhood-level interactions and the visibility of communal practices. In the simulation, once the local Subjective Norm ($SN$) parameter exceeded the critical threshold ($SN > 0.70$), the pressure to conform acted as a ``Social Norm Shield,'' effectively counteracting the natural decay of Attitude. This validates the assertion of \citet{Ceschi2021} regarding norm-based policies: residents continued to segregate not because of active policing, but because the high prevalence of compliance among neighbors created a decision-making automatism that overrode the Cost of Effort. By recognizing this graduation effect, the DRL agent was able to strategically treat funding as a catalyst rather than a permanent subsidy, creating a fiscally sustainable long-term policy.

\subsection{Challenges in High-Density Areas (The Poblacion Case)}

In stark contrast to the rapid stabilization of smaller communities, the high-density Barangay Poblacion required a sustained, resource-intensive intervention spanning five consecutive quarters (Q7--Q11) to achieve similar results. The analysis indicates that this resistance was driven by ``Density-Dependent'' factors; Poblacion's large population of 1,534 households meant that even when the DRL agent allocated 90\% of the total municipal budget to this single barangay, the resulting enforcement intensity only reached an initial effective value of approximately 0.38. This dilution effect confirms the ``Urban Resource Trap'' described by \citet{Villanueva2021}, where high population density dissolves the impact of standard monitoring mechanisms, allowing non-compliance to persist as the status quo despite significant absolute spending.

The simulation implies that for such high-density urban centers, standard budget allocation strategies are insufficient to break the initial resistance threshold. The data suggests the necessity of a specialized ``Shock and Awe'' policy, where funds are temporarily diverted entirely from Information, Education, and Communication (IEC) and Incentives to focus purely on Enforcement. This aligns with the observations of \citet{Nishimura2022} regarding the operational burdens of Local Government Units, highlighting that overcoming the inertia of large, anonymous urban populations requires a threshold of enforcement visibility that equal-distribution strategies can never mathematically achieve. The DRL agent's eventual success in Poblacion proves that overcoming this density barrier is possible, but only through the sequential concentration of resources to artificially induce the behavioral tipping point.

% =========================================
% SECTION 4.5: RESEARCH QUESTIONS
% =========================================
\section{Synthesis of Answers to Research Questions}

The empirical results generated by the simulation provide definitive answers to the three research questions posed at the outset of this study:

\textbf{RQ1: Impact of Behavioral Parameters.} The global sensitivity analysis and comparative scenario testing confirm that variations in household behavioral parameters fundamentally dictate policy stability. Specifically, the simulation revealed that the ``Cost of Effort'' is the dominant inhibitor of compliance, overriding high ``Attitude'' scores. This explains why previous IEC campaigns failed; increasing awareness (Attitude) does not compensate for the high logistical friction (PBC) in areas like Poblacion. Consequently, policies that do not physically reduce this cost—either through convenient infrastructure or high-intensity enforcement that makes non-compliance ``costlier''—are destined to fail, regardless of public awareness levels.

\textbf{RQ2: Optimal Resource Allocation Ratio.} The study determined that there is no single static ``golden ratio'' for resource allocation. Instead, the optimal allocation is dynamic and state-dependent. The DRL agent demonstrated that the most effective strategy involves a phased approach: an initial heavy concentration (up to 90\%) on Enforcement and Incentives to break resistance, followed by a gradual pivot toward IEC and maintenance spending once the ``tipping point'' of 70\% compliance is reached. This challenges the static 20\%/30\%/50\% splits often found in annual appropriation ordinances, proving that allocation ratios must evolve quarter-by-quarter based on real-time compliance data.

\textbf{RQ3: The Superior Dynamic Strategy.} The Heuristic-Guided PPO algorithm was unequivocally identified as the dynamic strategy yielding the highest cost-benefit ratio. While the ``Status Quo'' achieved only 12.5\% global compliance and ``Pure Enforcement'' collapsed to near-zero in key areas, the DRL-driven ``Sequential Saturation'' strategy achieved 100\% graduation of all barangays within 12 quarters without exceeding the budget cap. This confirms that the highest compliance per peso is achieved not by spending more, but by spending sequentially—concentrating capital to trigger social norms that subsequently sustain behavior for free.

% =========================================
% SECTION 4.6: IMPLICATIONS FOR GOVERNANCE
% =========================================
\section{Addressing the Municipality's Strategic Gaps}

The findings of this study provide a critical, data-driven framework for re-evaluating the solid waste management strategies of the Municipality of Bacolod. By transitioning from static administrative heuristics to dynamic, AI-optimized policy design, the results offer specific, actionable insights that directly address the municipality's chronic compliance deficits. This section discusses the practical implications of these findings.

For the Municipality of Bacolod, the most significant implication of this research is the invalidation of the ``Equal Distribution'' model. The current practice of spreading the limited annual budget of \textpeso 1,500,000 thinly across all seven barangays is mathematically proven to be the primary driver of failure in high-density areas. The simulation demonstrates that this approach dilutes enforcement intensity below the critical threshold required to alter behavior in urban centers like Barangay Poblacion. To address this, the LGU must adopt the ``Sequential Saturation'' strategy discovered by the DRL agent. This requires a political shift: rather than funding all barangays inadequately at the same time, the LGU should concentrate resources on one or two targets at a time—effectively ``besieging'' non-compliant zones until social norms take over—before rotating funds to the next priority area. This approach respects the strict budgetary ceiling while maximizing the per-peso impact of every intervention.

Furthermore, the study highlights the necessity of decoupling policy instruments. The finding that incentives fail in large populations due to budget dilution suggests that the LGU should restrict monetary rewards to smaller, manageable puroks or barangays where the per-capita value remains meaningful. Conversely, in high-density zones, the budget should be redirected toward high-visibility enforcement to overcome anonymity. This targeted differentiation allows the municipality to move away from a ``one-size-fits-all'' ordinance toward a precise, demographic-specific application of R.A. 9003.

% =========================================
% SECTION 4.7: SUMMARY OF FINDINGS
% =========================================
\section{Summary of Results and Discussion}

The comparative analysis and subsequent mechanistic investigation of the simulation results yield a definitive conclusion regarding the optimization of solid waste management policies in the Municipality of Bacolod. The primary finding is that the conventional ``Status Quo'' approach, characterized by an equitable distribution of resources across all barangays with a heavy emphasis on Information, Education, and Communication (IEC), is mathematically insufficient to achieve municipality-wide compliance. While this strategy is capable of maintaining order in smaller, socially cohesive communities, it consistently fails to overcome the high behavioral friction present in densely populated urban centers. This failure is not due to a lack of total funding, but rather a strategic error in resource dilution; spreading the budget thinly prevents the Local Government Unit (LGU) from ever reaching the critical enforcement intensity required to break the inertia of non-compliance in high-resistance zones.

Furthermore, the simulation demonstrates that single-instrument policies---whether purely punitive or purely incentive-based---are equally flawed when applied uniformly. Pure enforcement collapses due to its inability to sustain positive attitudes, while pure incentives fail due to the diminishing per-capita value of rewards in large populations. In contrast, the Deep Reinforcement Learning (DRL) agent's ``Sequential Saturation'' strategy provides a robust solution to this dilemma. By abandoning the heuristic of simultaneous equality in favor of sequential efficacy, the agent successfully leveraged the ``Graduation Effect,'' utilizing social norms to sustain compliance in stabilized areas while concentrating capital to besiege and conquer the most resistant urban targets. This confirms that the optimal policy for a resource-constrained LGU is dynamic and adaptive: a phased intervention that treats compliance not as a static maintenance task, but as a series of strategic tipping points to be systematically triggered.