% !TeX root = ../cs_thesis_main.tex

\chapter{Results \& Discussion}
This chapter presents the empirical parameterization of the simulation environment, the calibration of the Agent-Based Model (ABM) against real-world audits, and the comparative analysis of policy scenarios generated by the Deep Reinforcement Learning (DRL) agent.

\section{Analysis of System Inputs and Parameters}
The simulation’s environmental rules are strictly defined by the legal framework of Bacolod, Lanao del Norte. An analysis of Municipal Ordinance No. 2018-05 (Ecological Solid Waste Management Ordinance) and Appropriation Ordinance No. 01-2024 reveals three critical constraints that govern the RL agent's decision-making process.

\subsection{Legislative Constraints and Financial Bounds}
While Municipal Ordinance No. 2018-05 technically mandates a four-stream segregation system comprising biodegradable, residual, recyclable, and special waste, modeling the sorting accuracy of every specific waste stream introduces unnecessary computational noise. Consequently, the simulation abstracts this requirement into a binary classification state: a \textit{compliant state} represents the household’s effort to segregate according to the mandate, while a \textit{non-compliant state} denotes the practice of open dumping or commingled disposal. This abstraction focuses the model on the behavioral decision to adhere to the law—the compliance motivation—rather than the technical accuracy of waste sorting. Thus, a household is flagged as compliant within the simulation if their utility function exceeds the threshold for effort, representing adherence to the ordinance's intent.

Furthermore, the codified fines for non-compliance provided the exact values for the Reinforcement Learning agent’s negative reward signal. The ordinance stipulates penalties of Php 300.00 for the first offense, Php 500.00 for the second, and Php 1,000.00 for the third. These values were normalized within the simulation to define the magnitude of the punitive action available to the government agent. This ensures that the agent learns the trade-off between the political cost of enforcement and the monetary gain of compliance, rather than relying on arbitrarily assigned penalty values.

Finally, a review of the Municipal Appropriation Ordinance No. 01-2024 indicates that the Municipal Environment and Natural Resources Office operates under a finite fiscal ceiling. The total annual appropriation for the Solid Waste Management program, approximately Php 1.5 million, acts as a hard constraint in the simulation. This restricts the agent from executing policies that exceed a quarterly burn rate derived from this annual budget. Any policy trajectory that violates this constraint is terminated, thereby forcing the Proximal Policy Optimization (PPO) algorithm to optimize for cost-efficiency rather than attempting to solve the problem through unlimited spending.

\subsection{Heterogeneity of Barangay Operations and Behavior}
Data from the seven target barangays—Babalaya, Binuni, Esperanza, Demologan, Liangan East, Mati, and Poblacion—demonstrate that a ``one-size-fits-all'' policy is ineffective. The interview transcripts revealed significant heterogeneity in community behavior, which necessitated the initialization of diverse \texttt{BarangayAgent} profiles within the simulation.

The Initial State ($S_0$) of the simulation is not uniform. Interview data highlights a disparity in compliance culture: Barangay Liangan East exhibits a high baseline compliance (estimated at 60\%--70\%) attributed to active monitoring by barangay officials. In stark contrast, Barangay Poblacion operates at a ``critical failure'' state with compliance estimates as low as 2\%. This variance validates the use of a spatially explicit ABM, where the RL agent must learn different strategies for high-performing versus low-performing sectors.

Qualitative analysis of official transcripts identified the phenomenon of ``Behavioral Decay''—the tendency of residents to revert to non-segregation in the absence of consistent enforcement. Officials from Barangay Mati and Poblacion noted high rates of recidivism, whereas Barangay Binuni and Babalaya reported stronger adherence to social norms. These qualitative insights were quantified into a \texttt{memory\_decay\_rate} parameter ($\gamma$), where agents in ``Low Social Capital'' zones forget compliance incentives faster than those in ``High Social Capital'' zones.

Feedback from Liangan East and other rural barangays emphasized that economic barriers (e.g., the cost of purchasing distinct sacks or bins) hinder compliance for low-income households. To reflect this, the simulation incorporates an \texttt{Income\_Sensitivity} variable. The cost of compliance ($C_{\text{compliance}}$) weighs more heavily in the utility function of lower-income agents, making them less responsive to penalties and more responsive to subsidies or material support.

\section{Model Calibration and Validation}
Model calibration serves as a critical validation step to ensure that the simulation accurately reproduces the current real-world state of the Municipality of Bacolod prior to the introduction of optimization policies. This process necessitates reconciling the discrepancies observed between micro-level data gathered through barangay interviews and the macro-level reality established by the Municipal Environment and Natural Resources Office audit.

\subsection{Addressing Reporting Bias in Input Data}
A significant challenge identified during the initialization phase was the marked divergence between self-reported compliance rates and the observed municipal average. Aggregating the compliance rates claimed by barangay officials yielded a calculated municipal average ranging from approximately 40\% to 50\%, with specific localities such as Binuni and Demologan reporting rates as high as 95\% and 60\%, respectively. However, the municipal-wide audit conducted by the MENRO indicated an actual effective segregation rate of approximately 10\%. This discrepancy suggests the presence of social desirability bias in the interview data, where local officials may have overstated compliance to present a favorable political image. Consequently, the simulation utilizes the MENRO audit of 10\% as the ground truth target for calibration.

\subsection{Calibration Procedure}
To align the model with the empirical reality, a calibration factor was introduced to the household decision logic. The calibration process proceeded in two distinct stages. The initial simulation, termed the \textit{naive run}, was executed using the raw interview values; as anticipated, this yielded an inflated municipal compliance rate, confirming that the self-reported values represented optimistic upper bounds rather than accurate current states. Subsequently, the \textit{calibrated run} subjected the initial compliance values for each barangay to a global decay parameter until the simulation's aggregate output converged to the known 10\% municipal average. 

Specific adjustments, detailed in Table \ref{tab:calibration_params}, included lowering Binuni’s initial state from 95\% to 45\% to account for unreported open dumping, and reducing Liangan East from 60\% to 20\% to reflect intermittent segregation. Notably, the reported compliance for Poblacion remained at 2\%, as the low report aligned with MENRO data. The original reported values are retained within the model as the theoretical potential, representing the maximum achievable compliance under perfect enforcement, while the adjusted values serve as the actual initialization state for the simulation.

\begin{table}[htbp]
  \centering
  \caption{Calibration of Barangay Compliance Parameters}
  \label{tab:calibration_params}
  \begin{tabular}{@{}lccl@{}}
    \toprule
    \textbf{Barangay} & \textbf{Reported (\%)} & \textbf{Adjusted (\%)} & \textbf{Rationale for Adjustment} \\
    \midrule
    Binuni & 95 & 45 & Unreported open dumping practices \\
    Liangan East & 60 & 20 & Intermittent segregation consistency \\
    Poblacion & 2 & 2 & Valid; aligns with MENRO audit data \\
    \textit{Others} & \textit{Various} & $\sim$5--10 & Standardized to municipal average \\
    \bottomrule
  \end{tabular}
\end{table}