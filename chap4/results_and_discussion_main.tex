% !TeX root = ../cs_thesis_main.tex

\chapter{Results \& Discussion}
This chapter presents the empirical parameterization of the simulation environment, the calibration of the Agent-Based Model (ABM) against real-world audits, and the comparative analysis of policy scenarios generated by the Deep Reinforcement Learning (DRL) agent.

\section{Analysis of System Inputs and Parameters}
The validity of an Agent-Based Model (ABM) relies heavily on the empirical grounding of its initialization parameters. Before training the Deep Reinforcement Learning (DRL) agent, the real-world data gathered from the Municipality of Bacolod—comprising legislative documents and key informant interviews—were analyzed to define the state space, constraints, and reward structures of the simulation environment. This section details how qualitative field data was translated into quantitative simulation parameters.

\subsection{Legislative Constraints and Financial Bounds}
Based on the simulation parameters defined in the barangay configuration, the interview transcripts referenced in Appendix B, and the financial data provided by the Municipal Environment and Natural Resources Office (MENRO), this section details the legislative and financial constraints that define the solution space for the Agent-Based Model. The simulation environment is strictly bounded by the fiscal realities of the Municipality of Bacolod. Unlike theoretical models where resources may be infinite, this study incorporates hard constraints derived from Appropriation Ordinance No. 2024-01, ensuring that the Deep Reinforcement Learning (DRL) agent operates within the actual legal and financial framework of the Local Government Unit (LGU).

At the macro-environmental level, the MENRO acts as the primary resource allocator. The simulation enforces a strict Annual Budget Cap of \textpeso 1,500,000, a figure calibrated directly from the MENRO interview detailed in Appendix B.1. This creates a quarterly resource constraint of \textpeso 375,000. Under the ``Status Quo'' policy, these funds are typically distributed based on population density; however, the simulation reveals that a purely demographic distribution dilutes the impact in high-density areas while over-funding low-density areas. Furthermore, operational constraints are dictated by the regional economic reality. With a Daily Minimum Wage of \textpeso 400 based on Region X rates, the cost of human enforcement is significantly high. A single enforcer patrolling for a quarter consumes approximately \textpeso 24,000 to \textpeso 30,000. This financial reality creates a ``Personnel Trap'' where the budget is consumed entirely by salaries, leaving zero room for Information, Education, and Communication (IEC) campaigns.

\begin{table}[htbp]
    \centering
    \caption{Barangay Demographic and Financial Initialization Parameters}
    \label{tab:barangay_demographics}
    \small 
    \begin{tabular}{l r r c c c c}
        \toprule
        & & {Annual} & {Initial} & \multicolumn{3}{c}{{Income Profile (\%)}} \\
        \cmidrule(lr){5-7}
        {Barangay} & {Households} & {Budget (\textpeso)} & {Compliance} & {Low} & {Mid} & {High} \\
        \midrule
        Brgy Poblacion    & 1,534 & 200,000 & 2\%  & 70 & 25 & 5 \\
        Brgy Liangan East & 608   & 30,000  & 14\% & 40 & 40 & 20 \\
        Brgy Ezperanza    & 574   & 90,000  & 14\% & 20 & 50 & 30 \\
        Brgy Binuni       & 507   & 126,370 & 15\% & 50 & 30 & 20 \\
        Brgy Demologan    & 463   & 21,000  & 11\% & 80 & 15 & 5 \\
        Brgy Babalaya     & 171   & 15,000  & 14\% & 80 & 10 & 10 \\
        Brgy Mati         & 165   & 80,000  & 11\% & 90 & 5  & 5 \\
        \bottomrule
    \end{tabular}
    \vspace{1ex}
    
    \raggedright
    \footnotesize
    \textit{Note: Income profiles derived from Appendix B interviews. Low Income agents have higher price sensitivity ($\gamma$) to fines.}
\end{table}

The heterogeneity of the seven barangays creates a complex optimization landscape where each operates under unique financial bounds and legislative priorities. Barangay Poblacion represents an urban resource trap. Despite having the highest local budget of \textpeso 200,000, it suffers from the lowest per-capita spending power due to its massive population of 1,534 households. Defined as the ``Police State'' profile in Appendix B.8, Poblacion is legislatively bound to allocate 60\% of its budget to enforcement because the high density of commerce and residence necessitates a heavy presence of Barangay Tanods to maintain basic order. The simulation demonstrates that despite the high absolute budget, this 60\% enforcement lock leaves insufficient funds for IEC or incentives, resulting in a rigid compliance ceiling of approximately 2\%. Conversely, Barangay Binuni represents the simulation’s ``Best Case Scenario'' or wealthy anomaly. With a budget of \textpeso 126,370 for only 507 households, it enjoys a high per-capita budget. This fiscal surplus allows Binuni to adopt a balanced ``Rich \& Capable'' profile, as noted in Appendix B.3, splitting funds evenly between enforcement and incentives. This financial flexibility allows Binuni to overcome high behavioral friction, making it the consistent top performer in the Status Quo analysis.

Barangay Liangan East faces a critical mismatch that highlights the risks of incentive-heavy programs without sufficient funding. It attempts an ambitious program with a meager budget of \textpeso 30,000 for 608 households. Modeled after the ``Eco-brick'' project mentioned in Appendix B.6, Liangan East is legislatively biased toward incentives at 65\%, leaving only 25\% for enforcement. This creates a fragile system where the simulation shows that without LGU augmentation, the \textpeso 30,000 is quickly exhausted by incentive redemptions, leading to rapid program collapse. This validates the ``False Spike'' behavioral profile, where compliance rises briefly but crashes when funds run out. In contrast, Barangay Ezperanza represents the middle-class standard. With \textpeso 90,000 for 574 households, it serves as the median control group. Following a ``Consistent'' profile from Appendix B.5, it maintains a 50\% enforcement and 30\% incentive split, allowing it to maintain a baseline compliance of roughly 10\% without external intervention.

Barangay Mati acts as a small-scale surplus outlier with a high budget of \textpeso 80,000 for a tiny population of 165 households. Being 90\% agricultural as described in Appendix B.7, the legislative focus is on incentives rather than strict policing. The high budget-to-household ratio means Mati is highly responsive to financial injections; however, the simulation assigns a high decay rate to this profile, indicating that while farmers participate for rewards, the habit formation is volatile.

Finally, Barangays Babalaya and Demologan represent the poverty trap of the simulation, operating with budgets of \textpeso 15,000 and \textpeso 21,000 respectively. They are bound by the ``Survivor'' profile detailed in Appendices B.2 and B.4. With such limited funds, 85\% to 90\% of the budget is locked into enforcement, not as a strategic choice but a survival necessity. \textpeso 15,000 is barely enough to pay the honoraria of two Tanods for a quarter. Consequently, their IEC and incentive allocation is effectively zero. The simulation confirms that under the Status Quo, these barangays are mathematically incapable of improving compliance without external LGU subsidy. In summary, these legislative constraints create a ``Hierarchy of Difficulty'' for the DRL agent, forcing it to navigate the rigid enforcement spending of Poblacion, the budget deficits of Liangan East, and the poverty traps of Babalaya and Demologan rather than applying a uniform policy.

\subsection{Heterogeneity of Barangay Operations and Behavior}
A critical finding from the data gathering phase (Appendix B) is that the Municipality of Bacolod cannot be modeled as a monolithic entity. The simulation environment is characterized by profound heterogeneity across its seven constituent barangays, creating a non-convex optimization landscape for the Deep Reinforcement Learning (DRL) agent. A ``one-size-fits-all'' policy is mathematically destined to fail because the responsiveness of agents varies drastically depending on their local context. This section delineates the three primary dimensions of this heterogeneity—economic capacity, psychosocial drivers, and administrative inertia—which collectively define the unique state space for each locale.

The economic landscape of the municipality is sharply divided between agrarian subsistence zones and emerging commercial centers, a disparity that fundamentally alters the marginal utility of money within the simulation’s utility function. According to the income profiles calibrated from Appendix B, Barangays Babalaya and Demologan represent the municipality's economic floor, with 80\% of their households classified as low-income earners. In the Agent-Based Model, this high concentration of poverty assigns a higher gamma value to these agents, meaning that monetary fines are disproportionately punitive, yet the opportunity cost of time (compliance effort) is critically high due to subsistence labor. In stark contrast, Barangay Binuni exhibits a ``Wealthy Anomaly'' profile, comprising a mix of 50\% low-income, 30\% middle-income, and 20\% high-income households. This economic buffer reduces the friction of compliance, as wealthier households possess the resources to manage waste segregation without sacrificing essential labor time. Barangay Ezperanza serves as the emerging middle-class standard, with a balanced distribution that creates a predictable response to economic incentives. This heterogeneity implies that a uniform fine of \textpeso 500 is mathematically devastating for a household in Mati (90\% low-income/farmers) but merely a nuisance for a household in Binuni, forcing the DRL agent to weigh the equity of enforcement across these disparate economic zones.

\begin{table}[htbp]
    \centering
    \caption{Calibrated Behavioral Weights and Legislative Allocation Strategies}
    \label{tab:behavioral_parameters}
    \resizebox{\textwidth}{!}{ 
    \begin{tabular}{l c c c c c c c c}
        \toprule
        & \multicolumn{4}{c}{{Agent Behavior Parameters}} & & \multicolumn{3}{c}{{Budget Allocation (\%)}} \\
        \cmidrule(lr){2-5} \cmidrule(lr){7-9}
        {Barangay} & {Attitude ($w_a$)} & {Norms ($w_{sn}$)} & {Effort ($C_e$)} & {Decay ($\gamma$)} & & {Enf} & {Inc} & {IEC} \\
        \midrule
        Binuni       & 0.75 & 0.80 & 0.64 & 0.005 & & 40 & 40 & 20 \\
        Ezperanza    & 0.40 & 0.70 & 0.54 & 0.015 & & 50 & 30 & 20 \\
        Babalaya     & 0.80 & 0.90 & 0.62 & 0.030 & & 90 & 5  & 5 \\
        Liangan East & 0.65 & 0.60 & 0.58 & 0.050 & & 25 & 65 & 10 \\
        Poblacion    & 0.55 & 0.20 & 0.70 & 0.100 & & 60 & 20 & 20 \\
        Demologan    & 0.70 & 0.60 & 0.62 & 0.080 & & 85 & 10 & 5 \\
        Mati         & 0.60 & 0.50 & 0.60 & 0.100 & & 30 & 50 & 20 \\
        \bottomrule
    \end{tabular}
    }
    \vspace{1ex}
    
    \raggedright
    \footnotesize
    \textit{Note: Agent behavior parameters correspond to Theory of Planned Behavior (TPB) weights: $w_a$ (Attitude), $w_{sn}$ (Social Norms), $C_e$ (Perceived Cost of Effort), and $\gamma$ (Compliance Decay Rate). Budget allocation percentages reflect the legislative constraints under the Status Quo scenario.}
\end{table}

Beyond economics, the distinct psychosocial profiles of each barangay determine how they respond to policy interventions. The simulation encodes these differences through TPB weights—Attitude, Social Norms, and Perceived Behavioral Control—and a decay factor representing habit retention. Barangay Poblacion presents the most significant behavioral challenge; despite its size, it exhibits an ``Urban Isolation'' profile with the lowest Social Norm weight (0.20) and the highest Cost of Effort (0.75). This indicates that social pressure from neighbors is ineffective in the high-density commercial center, and the friction to comply is maximal. Conversely, Barangay Babalaya demonstrates the ``Tight-Knit Community'' paradox. It possesses the highest Social Norm weight (0.90) and Attitude (0.80), suggesting a population that deeply desires to comply due to strong community cohesion. However, this is offset by the rigid financial constraints mentioned previously. Furthermore, the model accounts for the persistence of education through variable decay rates. Barangay Mati, representing the agricultural sector, has the highest decay rate (0.10), implying that IEC campaigns are quickly forgotten as residents revert to traditional practices. In contrast, Barangay Binuni has a negligible decay rate (0.005), indicating that once a habit is formed, it is effectively permanent. This heterogeneity dictates that the DRL agent must deploy high-frequency reminders in Mati, while focusing on friction-reduction in Poblacion.

The final layer of heterogeneity lies in the entrenched spending habits of each local council, which sets the initial ``Action Space'' for the simulation. These allocation profiles are not random but are historical reflections of necessity and leadership preference. The ``Survivalist'' barangays—Babalaya (90\% Enforcement), Demologan (85\% Enforcement), and Poblacion (60\% Enforcement)—are locked into high enforcement spending. This is not due to a preference for policing, but a mathematical necessity: their budgets are so small relative to their populations (or strictly bound by density in Poblacion’s case) that paying the honoraria of Tanods consumes almost all available liquidity, leaving no room for alternative strategies. On the other end of the spectrum, Barangay Liangan East exhibits a ``Project-Based'' bias, allocating 65\% of its resources to Incentives. This reflects a specific local initiative, the Eco-brick exchange program, which prioritizes rewards over punishment. Barangay Mati similarly leans towards Incentives (50\%) to appeal to its agrarian constituents. These pre-set allocations create a rigid status quo; the challenge for the DRL agent is not just to optimize the budget, but to actively break these entrenched allocation patterns to free up resources for a more balanced, multi-objective approach.

\section{Model Calibration and Validation}
Model calibration serves as a critical validation step to ensure that the simulation accurately reproduces the current real-world state of the Municipality of Bacolod prior to the introduction of optimization policies. This process necessitates reconciling the discrepancies observed between micro-level data gathered through barangay interviews and the macro-level reality established by the Municipal Environment and Natural Resources Office audit.

\subsection{Addressing Reporting Bias in Input Data}
A significant challenge identified during the initialization phase was the marked divergence between self-reported compliance rates and the observed municipal average. Aggregating the compliance rates claimed by barangay officials yielded a calculated municipal average ranging from approximately 40\% to 50\%, with specific localities such as Binuni and Demologan reporting rates as high as 95\% and 60\%, respectively. However, the municipal-wide audit conducted by the MENRO indicated an actual effective segregation rate of approximately 10\%. This discrepancy suggests the presence of social desirability bias in the interview data, where local officials may have overstated compliance to present a favorable political image. Consequently, the simulation utilizes the MENRO audit of 10\% as the ground truth target for calibration.

\subsection{Calibration of Initial Compliance States}
A critical divergence was observed between the compliance rates self-reported by barangay officials during Key Informant Interviews (Appendix B) and the objective audit data provided by the Municipal Environment and Natural Resources Office (MENRO). While aggregate interview data suggested a municipal compliance rate exceeding 50\%, the MENRO’s waste characterization study confirmed a functional segregation rate of only 10\% to 15\%.

To ensure the Agent-Based Model reflects the true ``Status Quo,'' the initialization parameters in \texttt{barangay\_config.py} were calibrated to match the MENRO ground truth rather than the optimistic self-reports. This decision rests on three specific analytical pivots:

\begin{table}[htbp]
    \centering
    \caption{Comparison of Acquired Compliance Data vs. Calibrated Simulation Initialization}
    \label{tab:calibration_data}
    \small
    \begin{tabular}{l c c l}
        \toprule
        {Barangay} & {Acquired Data} & {Calibrated $S_0$} & {Primary Calibration Factor} \\
        & \textit{(Self-Reported)} & \textit{(Simulation Start)} & \\
        \midrule
        Brgy Babalaya     & 100\% & 14\% & Adjusted for lack of funding/monitoring \\
        Brgy Binuni       & 95\%  & 15\% & Decoupled awareness from actual practice \\
        Brgy Mati         & 70\%  & 11\% & Adjusted for high habit decay rate \\
        Brgy Liangan East & 65\%  & 14\% & Removed temporary project-based spikes \\
        Brgy Demologan    & 60\%  & 11\% & Standardized to municipal baseline \\
        Brgy Ezperanza    & 20\%  & 14\% & Aligned with middle-class control group \\
        Brgy Poblacion    & 2\%   & 2\%  & {Anchor Point} (Aligned with Audit) \\
        \bottomrule
    \end{tabular}
    \vspace{1ex}
    
    \raggedright
    \footnotesize
    \textit{Note: "Acquired Data" refers to unverified compliance rates reported during Key Informant Interviews (Appendix B). "Calibrated $S_0$" refers to the initialized compliance state in the Agent-Based Model, tuned to match the aggregate municipal segregation rate of $\approx$12\% verified by MENRO.}
\end{table}

Barangay Binuni officials reported a near-perfect compliance rate of 95\% (Appendix B.3). However, initializing the simulation at this level resulted in immediate global compliance exceeding 40\%, rendering the RL agent’s optimization tasks trivial. The simulation logic posits that the reported 95\% reflects \textit{awareness} rather than \textit{practice}. Consequently, Binuni’s initial state was calibrated to 15\%. While this appears to be a drastic reduction, it remains the highest initialization value in the system, preserving Binuni's relative status as the ``best performer'' while aligning its absolute magnitude with the municipal reality.

Barangay Liangan East reported 60\% compliance driven by an Eco-brick initiative. However, longitudinal analysis suggests that project-based compliance is transient. To model the ``Status Quo'' before the DRL agent's intervention, the system initializes Liangan East at 14\%. This reflects the baseline habituated behavior of the population when specific project funds are not actively being disbursed, preventing the simulation from starting at an artificial peak.

The only data point where self-reporting matched the MENRO audit was Barangay Poblacion, which reported a 2\% compliance rate. This convergence serves as the \textit{Anchor Point} for the simulation. Because the qualitative data and quantitative audit agreed on the lower bound, the simulation initializes Poblacion at 2\%. This validates the ``Urban Resource Trap'' theory, where high population density and anonymity dissolve social pressure, creating a distinct legislative challenge that the DRL agent must solve through enforcement rather than social norms.

By calibrating the initial states to this conservative baseline (2\% to 15\%), the simulation stabilizes at a Global Average Compliance of $\approx$12.5\% at $t=0$. This provides a valid, non-inflated baseline against which the DRL agent's policy improvements can be accurately measured.