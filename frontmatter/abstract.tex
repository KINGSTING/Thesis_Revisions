\begin{abstract}

The ongoing struggle with low household compliance regarding solid waste segregation rules, as mandated by the Philippine Ecological Solid Waste Management Act (RA 9003), significantly constrains the effectiveness of municipal solid waste management. While Local Government Units (LGUs) employ various incentives and penalties, these strategies often fail to account for the heterogeneous behaviors of households or their long-term economic impact. This study proposes a comprehensive Agent-Based Model (ABM) and Deep Reinforcement Learning (DRL) framework to identify the optimal policy mix for maximizing segregation compliance in the Municipality of Bacolod, Lanao del Norte. The ABM simulates household decision-making by integrating the Theory of Planned Behavior (Attitude, Subjective Norms, and Perceived Behavioral Control) with socio-demographic factors and policy constraints. The model is parameterized using empirical behavioral data and secondary records from the Philippine Statistics Authority (PSA) and the local LGU. A DRL algorithm is then deployed to enable the LGU agent to autonomously discover the most cost-effective policy strategy—whether purely incentive-based, punitive, educational, or a hybrid approach. The ultimate goal is to develop a validated computational tool and a set of data-driven recommendations, providing LGUs with a robust, evidence-based method for designing policies that enhance compliance while optimizing public funds.

\vspace{1cm} 

\noindent \textbf{Keywords}---\textit{Household Waste Segregation, Agent-Based Modeling, Deep Reinforcement Learning, Theory of Planned Behavior, Policy Optimization, Local Governance}

\end{abstract}