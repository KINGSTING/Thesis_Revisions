% !TeX root = ../cs_thesis_main.tex
\chapter{Review of Related Literature}

This chapter synthesizes the body of scholarly work that forms the foundation for this research, which aims to optimize solid waste management (SWM) policies for a Philippine Local Government Unit (LGU) using an Agent-Based Modeling (ABM) and Deep Reinforcement Learning (DRL) approach. The review is organized thematically. It begins by examining the specific context of SWM in the Philippines under the Ecological Solid Waste Management Act \citep{RA9003}. It then explores the theoretical underpinnings of pro-environmental behavior and the policy instruments used to influence it. Subsequently, it delves into the computational methodologies of ABM and DRL, highlighting their applications in environmental management and their synergistic potential. The chapter concludes by synthesizing these areas to clearly identify the research gap that this thesis aims to address.

\section{Solid Waste Management in the Philippines}
\setlength{\parindent}{2em}

The national framework for waste management is defined by the Ecological Solid Waste Management Act of 2000 \citep{RA9003}, which mandates source segregation, recycling, and the establishment of Materials Recovery Facilities (MRFs). However, \citet{Coracero2021} and \citet{Salsabila2021} confirm that the implementation of R.A. 9003 remains sub-optimal and unsustainable across Philippine LGUs, necessitating a strategic, computational approach to bridge the gap between policy and practice.

\subsection{Systemic and Budgetary Constraints on LGUs}
\setlength{\parindent}{2em}

The primary challenge in implementing R.A. 9003 lies in the significant operational and institutional burden placed on Local Government Units (LGUs), which function as the chief implementers of the law \citep{Nishimura2022}. This burden is most evident in the financial and technical overload that municipalities face. Solid Waste Management (SWM) consistently constitutes a high financial drain on municipal budgets. This cost is compounded by systemic deficiencies, such as a scarcity of compliant sanitary landfills and a chronic lack of funding for local initiatives. Together, these issues often lead to a form of institutional failure that fundamentally weakens the law's effectiveness \citep{Ibanez2022, Santos2025}. Overcoming these deep-seated structural and financial constraints requires LGU officials to demonstrate considerable political initiative to improve performance \citep{Nishimura2022}.

Beyond financial hurdles, volatile enforcement undermines policy credibility. Assessments across the Philippines consistently report that while local ordinances are enacted, their enforcement on the ground is inconsistent, weak, or flagging \citep{Dalugdog2021, Sagodaquil2023}. \citet{Yazawa2025} characterize this as the critical ``Act vs. Reality'' gap, where barangay-level practices deviate significantly from the text of RA 9003. This observation is corroborated by \citet{Villanueva2021} and \citet{ApostolJamoralin2024}, whose assessments confirm that without strict monitoring mechanisms and political will, the ``status quo'' of non-compliance persists despite clear legal frameworks. This lack of strict, sustained implementation erodes the punitive element of the policy, decreasing its credibility and, consequently, its effectiveness as a deterrent \citep{Badua2022}. This creates a critical optimization problem: a model must learn the most cost-effective threshold of enforcement required to build and maintain policy credibility, all while operating within a realistic and fixed budget constraint.

Finally, the governance structure of the Philippines complicates policy implementation. The Barangay, the basic political unit, serves as the primary level for both planning and implementing R.A. 9003 \citep{Delena2025}. This multi-level structure necessitates a modeling approach, such as an Agent-Based Model (ABM), that can accurately simulate the flow of policy mandates from the Municipal LGU down to the Barangay level. \citet{Florida2023} emphasize that the performance rating of these barangays is heavily dependent on localized administrative capability, which varies significantly. Furthermore, specific geographical contexts exacerbate these governance challenges. \citet{DelRosario2023} highlights that coastal municipalities—similar to the Municipality of Bacolod—face distinct logistical burdens in preventing marine debris that inland models often fail to address. Such a model must therefore account for the significant demographic differences and resource variations that exist between individual barangays \citep{Brugiere2022}.

\subsection{Behavioral and Policy Intervention}
\setlength{\parindent}{2em}

A persistent gap between household awareness and actual practice highlights the limitations of simple mandates, underscoring the need for sophisticated behavioral interventions \citep{Catiil2025}. Studies across various Philippine cities demonstrate that a high level of resident awareness of R.A. 9003 often fails to translate into consistent, proper segregation \citep{Abordo2025}. This dissonance is quantified by \citet{Paigalan2025}, who observed that while residents in Northern Mindanao exhibited positive attitudes toward waste separation, improper practices such as open waste burning remained prevalent. This widespread phenomenon justifies the application of the Theory of Planned Behavior (TPB), which posits that external factors—mainly Subjective Norms (community perceptions) and Perceived Behavioral Control (the perceived ease or difficulty of segregating)—can override an individual's positive intentions.

To address this, the literature strongly validates the need for multi-pronged interventions that combine informational and community-based tools \citep{Trushna2024}. Qualitative inquiries by \citet{Espino2025} reveal that resident non-compliance is often driven by genuine frustration with irregular collection services, while \citet{Carpio2025} frame this negligence through a green criminology perspective, highlighting the normalization of minor environmental offenses. To counter this, \citet{Camarillo2021} emphasize the need for participatory governance to build trust. Furthermore, \citet{Collado2024} demonstrate through the SURWEM project that targeted educational interventions can significantly raise awareness, though they note that awareness alone does not guarantee sustained behavioral change without structural support. This evidence directly supports modeling Educational Campaigns as a dynamic LGU expenditure within the ABM, specifically designed to increase the Attitude and Subjective Norms scores of household agents over time.

Furthermore, policy design must account for heterogeneity and equity. Financial penalties, such as fines, are known to disproportionately affect low-income groups, making the policy regressive and unjust \citep{Badua2022}. Therefore, the Reinforcement Learning agent must be tasked with optimizing not only for cost-efficiency but also for policy equity. To achieve this, the model must simulate heterogeneous agents whose sensitivity to both positive incentives and negative penalties varies based on their synthesized socio-economic profile.

\section{Policy Behavioral Interventions}
\setlength{\parindent}{2em}

To enforce waste segregation, Local Government Units (LGUs) worldwide rely on a policy mix of economic incentives, regulatory penalties, and educational campaigns. The critical task for policymakers is to find the optimal combination and intensity of these levers—one that maximizes public compliance and segregation rates without exceeding finite public budgets \citep{HeFu2021, Torkayesh2021}. This presents a complex optimization problem, as these policy levers are not only budget-dependent but also highly interconnected. This research, therefore, models the LGU's strategic choice across three such interconnected, budget-dependent policy levers.

\subsection{Economic and Regulatory Levers}
\setlength{\parindent}{2em}

Economic instruments, frequently framed as reward-penalty schemes, are among the most powerful direct drivers of compliance because they immediately alter the financial cost-benefit analysis of the household segregation decision \citep{MuZhang2021, Zhao2022}. While \citet{Cheng2022} focused on construction waste and \citet{Wang2023} analyzed closed-loop supply chains, both concluded that government-led incentive-punishment mechanisms are essential for rationalizing waste reduction behavior. The literature consistently confirms that a hybrid approach—one combining both incentives and fines—is more effective than relying on either instrument alone \citep{Chen2023, MuZhang2021}. \citet{Rathore2021} further argue that identifying these suitable motivational mechanisms is critical for the success of any reverse logistics or collection system. This dual strategy effectively leverages both the psychological gain associated with rewards and the powerful aversion to loss associated with penalties.

A critical complication, however, is that the effectiveness of these financial policies is not uniform. The impact of both incentives and penalties varies significantly based on household heterogeneity \citep{Chen2023, Zheng2022}. For instance, low-income households tend to be more sensitive to the disutility of fines, while the perceived benefit of an incentive may be modulated by a household's education level or the perceived complexity of the program \citep{Zheng2022}. This finding directly justifies the methodological necessity of an Agent-Based Model (ABM), which can model distinct household agents whose sensitivity to the LGU’s financial actions is weighted by their synthetic socio-economic profiles.

Finally, these economic policies must be optimized not only for effectiveness but also for cost and robustness against uncertainty \citep{Gentile2022}. Robust optimization models in waste management explicitly seek to minimize the ``price of robustness''—that is, the extra cost incurred to protect the system against uncertain parameters, such as fluctuating waste volumes \citep{Gentile2022}. This concept directly mirrors the LGU’s practical constraint: the need to find a policy balance that avoids over-spending on enforcement or incentive schemes that, while effective, may yield diminishing returns \citep{Gentile2022}.

\subsection{Educational and Behavioral Levers}
\setlength{\parindent}{2em}

Educational strategies represent a critical, non-monetary intervention essential for achieving the long-term sustainability of Solid Waste Management (SWM) programs. Their fundamental value lies in their ability to address the root behavioral challenges that often undermine the success of technical or financial policies alone \citep{Moeini2023}. Complementing traditional education, \citet{LoanBalanay2023} advocate for the application of Nudge Theory, suggesting that subtle architectural changes and positive reinforcements can be as effective as strict mandates in reinforcing waste separation habits. For a Local Government Unit (LGU), educational campaigns are the primary tool for dynamically influencing household behavior, particularly by targeting the core constructs of the Theory of Planned Behavior (TPB). 

For a Local Government Unit (LGU), educational campaigns are the primary tool for dynamically influencing household behavior, particularly by targeting the core constructs of the Theory of Planned Behavior (TPB). These campaigns can cultivate a more positive Attitude toward segregation by clearly conveying its environmental importance. They also enhance Perceived Behavioral Control (PBC) by providing specific, practical knowledge on how to segregate properly, thereby increasing residents' confidence that the action is feasible. Furthermore, education reinforces Subjective Norms (SN) by increasing social awareness and fostering a community-wide expectation of compliance \citep{Vorobeva2022}.

The impact of these educational efforts extends beyond mere awareness, playing a significant role in the adoption of new systems. Studies show that ``soft'' behavioral factors, such as an established pro-environmental behavior (PEB) and a sense of empowerment, are crucial drivers for household participation. Notably, this influence persists even when financial incentives are part of the policy mix \citep{Vorobeva2022}. This supports a modeling approach where investment in education—by enhancing these foundational behavioral factors—improves the general willingness of agents to participate in and comply with LGU programs, complementing other interventions.

An integrated Agent-Based Modeling (ABM) framework is particularly well-suited for simulating these complex policy interactions. ABM has been successfully validated in prior research for its ability to simulate community-level behavioral responses and visualize compliance patterns under multiple incentive policies \citep{Ma2023}. This established precedent provides confidence that the model can accurately capture and predict the complex, emergent results of a hybrid policy that combines both educational interventions and financial incentives.

\section{Theoretical Foundations of Behavioral Modeling}
Understanding household decision-making is the foundational requirement for designing effective and cost-efficient waste segregation policies. The core of this research's Agent-Based Model (ABM) relies on extending established behavioral theory, primarily the Theory of Planned Behavior (TPB), to operate within a complex, stochastic environment.

\subsection{The TPB Framework and the Utility Function}
\setlength{\parindent}{2em}

The foundational ``brain'' of each household agent in the model is built upon the Theory of Planned Behavior (TPB), which serves as the dominant psychological framework for explaining and predicting pro-environmental behaviors (PEB), such as waste segregation. Drawing from the study of \citet{Ceschi2021}, this theory posits that an individual's intention to perform a behavior—and consequently the likelihood of the behavior itself—is determined by four core cognitive constructs: Attitude, Subjective Norms, and Perceived Behavioral Control.

The validity of utilizing the TPB as the psychological core of the utility function is well-supported in recent literature. A systematic review by \citet{Taraghi2025} found that Agent-Based Modeling (ABM) researchers frequently adopt the TPB to simulate pro-environmental behaviors, employing internal and external control variables to operationalize the theory's concepts. This is further supported by \citet{Ceschi2021}, who confirmed the validity of operationalizing these constructs---Attitude, Subjective Norms, and PBC---as weighted internal decision drivers.

To mathematically represent this decision-making process, the model employs a linear utility function where the agent's utility to segregate ($U_{\mathrm{segregate}}$) is the sum of these weighted psychological factors and a stochastic term:
\begin{equation}
    U_{\mathrm{segregate}} = (w_A A + w_{SN} SN + w_{PBC} PBC) + \epsilon
    \label{eq:behavior_equation}
\end{equation}
In this equation, the coefficients (\textit{w}) represent the relative weight or importance the agent assigns to each psychological factor. Crucially, the $\epsilon$ (epsilon) term accounts for the inherent stochasticity (randomness) and unobserved factors present in all human decision-making \citep{Chen2023, Zheng2022}. As individuals do not always act with perfect rationality, $\epsilon$ represents ``noise''---such as haste, forgetfulness, or momentary influences not captured by the primary variables. This stochastic element is essential for model realism, as real-world Solid Waste Management (SWM) systems are characterized by deep uncertainty. Therefore, embracing this randomness is required to produce valid insights into system behavior \citep{Akbarpour2021, Subedi2025}.

\section{ABM in Environmental Management}
\setlength{\parindent}{2em}

Agent-Based Modeling (ABM) is a powerful computational method for simulating the actions and interactions of autonomous agents within a defined environment. Serving as a ``virtual laboratory'' \citep{deSouza2021}, ABM is uniquely suited for modeling complex socio-environmental systems (SES) where macro-level outcomes, such as community compliance, emerge from micro-level behaviors and interactions \citep{Brugiere2022}. This capacity is paramount for analyzing municipal solid waste (MSW), where system-wide compliance results directly from the cumulative decisions made at the household level \citep{Fontaine2024}.

The necessity of ABM in this research is rooted in its ability to model heterogeneity and adaptive behavior. While other methodologies like System Dynamics (SD) are effective for analyzing aggregate stocks and flows of plastic waste \citep{Dhanshyam2021}, and systematic reviews link macro-population growth to generation rates \citep{Eltanal2025}, these approaches fail to capture individual decision-making. Unlike traditional system dynamics models, ABM captures the diversity of households, representing them as agents with distinct socio-demographic factors and psychological profiles, as defined by the Theory of Planned Behavior (TPB) \citep{Fontaine2024}. This allows the simulation of non-linear and adaptable responses to policy changes---a critical feature since household behaviors are not static. Furthermore, ABM's capacity to integrate advanced computational techniques, such as incorporating machine learning (ML) classifiers into agent decision logic, enhances behavioral realism beyond static heuristics and improves the accuracy of predicted policy outcomes \citep{Jimenez2025}.

\section{Deep Reinforcement Learning for Optimization}
\setlength{\parindent}{2em}

While the Agent-Based Model (ABM) serves as the simulation environment, Deep Reinforcement Learning (DRL) is the critical methodology required to autonomously discover optimal, budget-constrained policies within that complex system \citep{ZhengAI2022}. DRL integrates the decision-making framework of Reinforcement Learning with the representation learning capabilities of Deep Neural Networks (DNNs). In this setup, the Municipal LGU agent utilizes a neural network to approximate the optimal policy, learning to map high-dimensional state inputs---such as the heterogeneous psychological states of thousands of households---to precise adaptive decisions \citep{Hertweck2023}. This integration is essential because traditional tabular RL methods cannot scale to the massive, continuous state spaces generated by complex socio-environmental models \citep{Mousavi2021}.

To enable this autonomous optimization, the literature supports framing the ABM simulation not merely as a model, but as a stochastic environment formally defined as a Markov Decision Process (MDP). In this hybrid architecture, the ABM functions as a high-fidelity data generator, providing the state representation ($S_t$)---encapsulating household compliance levels and budget statuses---and processing the policy agent's actions ($A_t$) to generate the next state ($S_{t+1}$) and a corresponding reward signal ($R_t$). Studies in urban resource management \citep{Rajesh2025} and smart city logistics \citep{HaMinh2025} demonstrate that this approach provides the ``experience replay'' data required to train Deep RL agents in the absence of pre-existing datasets \citep{Kompella2020}.

Current applications of DRL in waste management remain predominantly focused on technical, industrial, or hardware-based optimization. For instance, recent advancements have extensively demonstrated the efficacy of DRL in automating waste classification and detection. \citet{Duhayyim2022, Khan2024} utilized Deep Q-Networks (DQN) and Mask R-CNN to automate the complex visual task of waste object detection, achieving high accuracy in segregating recyclables at the processing stage. Similarly, \citet{Udayakumar2023} integrated Improved Particle Swarm Optimization (IPSO) with MobileNetV2 to enhance the precision of waste categorization in smart city frameworks. Beyond classification, DRL has been applied to control industrial processes, such as waste biorefining \citep{Gao2024} and plant machinery optimization \citep{Kumar2022}. In logistics, stochastic optimization has been used for vehicle routing \citep{Khallaf2024, Akbarpour2021} and reverse logistics \citep{Karagoz2022}.

\section{Synthesis and Identification of the Research Gap}
\setlength{\parindent}{2em}

A comprehensive review of the literature indicates a critical implementation deficit in the Philippine Solid Waste Management (SWM) system. This paralysis, centered on Republic Act 9003, is consistently attributed to weak, inconsistent enforcement and chronic budgetary constraints within Local Government Units (LGUs) \citep{Santos2025, Sagodaquil2023}. Furthermore, studies highlight a significant gap between high public awareness of SWM and low actual compliance. This suggests that effective policy must move beyond simple enforcement, requiring a strategic blend of hybrid reward-penalty schemes and investments in non-monetary levers, such as education, to address core behavioral flaws \citep{Catiil2025, Chen2023}.

Addressing this behavioral component is complicated by the highly heterogeneous and non-linear nature of household decision-making. \citet{Olawade2024} identify this integration of Artificial Intelligence into waste management as a necessary ``paradigm shift,'' moving from reactive systems to smart, predictive management. The sheer complexity of these interacting variables strongly indicates that an agent-based approach is necessary to capture this dynamic behavior accurately.

\begin{longtable}{L{3.5cm} c c c L{5cm}}
    \caption{Thematic Comparison of Related Literature} \label{tab:lit_gap_analysis} \\
    \toprule
    \textbf{Category} & \textbf{ABM} & \textbf{DRL} & \textbf{Policy} & \textbf{Gap / Limitation} \\
    \midrule
    \endfirsthead
    \multicolumn{5}{c}{{\bfseries \tablename\ \thetable{} -- continued from previous page}} \\
    \toprule
    \textbf{Category} & \textbf{ABM} & \textbf{DRL} & \textbf{Policy} & \textbf{Gap / Limitation} \\
    \midrule
    \endhead
    \bottomrule
    \multicolumn{5}{r}{{Continued on next page}} \\ 
    \endfoot
    \bottomrule
    \endlastfoot

    % ROW 1
    Descriptive Studies \par (Abordo et al., 2025) & \xmark & \xmark & \xmark & Audits past compliance but cannot predict future policy outcomes. \\ 
    \midrule

    % ROW 2
    Traditional ABM \par (Bire et al., 2025) & \cmark & \xmark & \cmark & Relies on manual scenario testing; lacks autonomous AI optimization. \\ 
    \midrule

    % ROW 3
    Logistics Optimization \par (Akbarpour et al., 2021) & \xmark & \xmark & \xmark & Optimizes fleet routes, ignoring complex household behavior. \\ 
    \midrule

    % ROW 4
    AI \& RL (Hardware) \par (\citet{Dey2025}) & \xmark & \cmark & \xmark & Focuses on smart bins and robotics rather than governance policy. \\ 
    \midrule

    % ROW 5
    Qualitative Reviews \par (Santos et al., 2025) & \xmark & \xmark & \xmark & Offers no quantitative mechanism to optimize specific policy parameters. \\ 
    \midrule

    % ROW 6
    Proposed Study \par (Bansao \& Lumingkit) & \cmark & \cmark & \cmark & First study to use ABM-DRL to optimize Philippine LGU ordinance parameters. \\ 

\end{longtable}

\noindent{\textbf{The Critical Research Gap}}

While the academic literature validates the individual utility of Agent-Based Modeling (ABM) for simulating social systems and Deep Reinforcement Learning (DRL) for optimization, a critical research gap persists at their intersection, specifically within the domain of adaptive public policy. To date, no existing study has developed an integrated framework where a governing agent, such as a Local Government Unit (LGU), operates under a strict municipal budget to autonomously learn the optimal dynamic allocation of funds. This gap is particularly evident in the context of Philippine solid waste management, where the allocation of resources across three distinct strategic instruments---punitive enforcement, monetary incentives, and behavioral education---has not been mathematically optimized to maximize long-term household segregation compliance in a resource-constrained environment.

Current research in this domain remains bifurcated. On one hand, existing ABM studies in waste management primarily engage in static scenario testing. This approach typically involves comparing the outcomes of a few pre-defined policy mixes, rather than allowing the governing agent to autonomously discover the optimal, and potentially evolving, combination of interventions. On the other hand, optimization studies using DRL tend to focus exclusively on technical or logistical networks. These technical models generally fail to incorporate the dynamic social feedback loops necessary for understanding behavioral compliance and often neglect the LGU's strategic choice to allocate finite resources toward non-financial levers, such as educational campaigns.

\begin{equation}
    U_{\mathrm{segregate}} = (w_A(t) A) + (w_{SN}(t) SN_{\mathrm{local}}) + (w_{PBC}(t) PBC_{\mathrm{infra}}) - C_{\mathrm{Net}} + \epsilon
    \label{eq:proposed_utility_function}
\end{equation}

This study proposes a structural modification to the conventional Theory of Planned Behavior (TPB) framework, advancing beyond the static behavioral models typified by \citet{Ceschi2021} by introducing a dynamic, time-variant architecture. Central to this design is the integration of an explicit external policy term, $C_{\mathrm{Net}}$, alongside the standard internal psychological constructs. Unlike state-of-the-art models that treat government interventions as fixed background variables, this framework renders the utility function dynamic through time-variant weights ($w(t)$). Specifically, the model simulates non-linear behavioral evolution: the weight of Attitude ($w_A(t)$) follows a decay model to simulate ``public forgetting'' in the absence of reinforcement, requiring continuous IEC investment to maintain. Furthermore, the model incorporates ``psychological reactance,'' where excessive enforcement pressure inversely affects the agent's attitude, acknowledging the resistance to coercion often overlooked in standard simulations.

The novelty of this approach lies in the strategic decoupling of the agent's internal psychological state from external policy levers. By isolating the policy term within the utility equation, this research transforms the TPB from a purely descriptive tool into a computational interface for a Deep Reinforcement Learning (DRL) agent. This modification addresses a critical gap in current literature by modeling the LGU not merely as a static administrator, but as a ``strategic learner.'' Consequently, the RL agent can mathematically manipulate the external utility derived from compliance---balancing incentives, enforcement, and information \& educational campaigns ($IEC$) against budget constraints---without invalidating the agent's internal psychological nature. This coupled ABM-DRL framework moves beyond describing why agents segregate to dynamically optimizing how an LGU can induce segregation, offering a cost-effective decision-support tool for implementing R.A. 9003.